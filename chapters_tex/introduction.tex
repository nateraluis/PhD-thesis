\chapter{Introduction: Cities and networks}
%Why study cities? 

\epigraph{Cities have the capability of providing something for everybody, only because, and only when, they are created by everybody.}{Jane Jacobs, \textit{The Death and Life of Great American Cities} (1961, p. 238)}

As a complex system, a city offers a fertile study ground from multiple perspectives. Urban studies, from sociological perspectives to more technical ones such as engineering, have engaged in analyzing different aspects of urban life. Due to this diversity and overlapping approaches, the ``science of cities'' is inherently interdisciplinary.

\textcolor{red}{In cities, interdisciplinary goes beyond the application of tools and methods borrowed from one discipline, and their application to study a different object. Indeed, when we think and engage in the analysis of urban phenomena, we encounter a multitude of approaches without a single dominant one. When studying cities one can start taking apart its pieces and analyze them separately. Take for example buildings, roads, and functions, all of them can be studied without taking into consideration the rest of the pieces. However, following this approach misses the interactions between the pieces. Thus, to have a complete view of those interactions, and a better understanding of cities, we can study them as a system. To do so, we can use networks.}

\section{From architecture to network science, a personal journey}

As an architect my first approach to study cities was from the buildings' perspective, understanding how, by building, we delimit and shape spaces that model experiences in the city~\cite{gehl1971life}. These buildings are fundamental to define public spaces, mobility infrastructures, and even services that enable us to inhabit the \textit{ville}~\cite{sennett2018building}. The way we shape the city has an influence in how we inhabit it. Thus, understanding its infrastructures is fundamental to understand and plan better cities for an increasingly complex future. Especially, when tackling urban mobility challenges, the way we plan, build and use mobility infrastructures is fundamentally entangled with the livability of our cities.

\textcolor{red}{When thinking about urban infrastructures, my first approach was to use drawings and blueprints. These tools are useful to understand, and plan urban mobility. They provide a way to move from the physical structure of cities, to a more concrete way to represent them. These maps and blueprints provide a useful way to abstract the reality and make it more manageable. Of course, one has to be cautious when building such maps, at the end they are a representation of the territory, not the territory itself~\cite{borges1961hacedor}. Maps and blueprints provided me with an overview of cities and tools to move between scales. However, I was still missing tools to capture their intricacies and relations between multiple elements. Tus, I became interested in public policy as a mechanism to promote and incorporate changes in cities.}

After working in designing public policies to promote bicycle infrastructure in my home-city's government, I became interested in getting to understand the relation between different transportation modes, and how cities and their mobility infrastructures could be studied using large scale data systematically. This curiosity led me to the complex systems field, and the use of network science to study cities.

\section{Network science and its applications to cities}

\textcolor{red}{Cities are more that the sum of its parts. It is in the interactions between its different pieces and components, that one can find the richness of cities. Thus, modelling them as complex systems allow us to capture its interplays. In a mostly urbanized world, capturing these relations is fundamental to not only understand the cities, but to plan better and more human ones.}

%Studying cities as networks
\textcolor{red}{The complex system framework, has enabled a different research optic of cities. Such as modelling them as fractals to characterize it shape~\cite{batty1996preliminary}. Analyzing the continuous shape of the city has been a key element to understand its evolution, and how urbanized areas grow~\cite{makse1995growth}.}

\textcolor{red}{The complex system of urban transport infrastructure is shaping the dynamics of human mobility, a network embedded in space with special properties. The availability of data sets on urban infrastructure networks has enabled first insights on the topological constraints on human movements. Large-scale mobile phone data sets, on the other hand, have been used as proxies to understand mobility patterns in cities~\cite{gonzalez2008understanding}, finding predictable human mobility motifs related to socio-economic activities in cities. The use of such massive data sets has also provided insights into the possibilities of shareability, and into the efficient allocation of space for different types of transportation.}


Topics to cover:
\begin{itemize}
    \item Networks
    \item Space constrained networks. Cities as a network have different properties than other complex networks.
    \item The always evolving city. We cannot predict the future, we can build it. Tools to understand the present and build a sustainable future.
\end{itemize}
% While predicting how the future city will be is an impossible task, understanding how it has evolve and what is its state of development is possible. This understanding 


\luis{I still have to write about the thesis, from where does it start (complex systems and the study of cities) to the main contributions. I anticipate two paragraphs for the complex systems and cities, and two more paragraphs for the contributions/overview.}

\section{Aim and structure of the thesis}

The primary aim of this thesis is to contribute to the better understanding of the structural properties of multimodal transportation networks in urban areas. For that, we build on the tools and methods of network science to analyze the underlying complexity of urban mobility infrastructure. 

In the first part of this thesis we focus the attention on the multimodal infrastructure networks, first providing a review of previous works, and then giving an original contribution to the analysis of multilayer transportation networks. In the second part, we focus the attention to specific layers, first bicycle, then pedestrian infrastructure.

This thesis is structured as follows:

\begin{itemize}
    \item Chapter~\ref{ch:litReview}: We provide a review of previous works on multimodal transportation and mobility research from a complex systems' perspective. First we focus on the infrastructure and measures to quantify it, then on the mobility dynamics on top of these infrastructures, and finally we offer a review of available datasets and tools to analyze this specific type of networks.
    \item Chapter~\ref{ch:OverlapCensus}: We make an original contribution to the field by proposing a method to extract the multimodal profile from a city's multiplex transport network. We apply our methods to fifteen cities, finding clusters of cities with similar multimodal infrastructure.
    \item Chapter~\ref{ch:BikeGrowth}: We focus our attention in the bicycle layer of the multimodal network and propose algorithmic approaches to improve its connectivity. We find that focalized investment has the potential to rapidly improve the connectedness and directness of the bicycle infrastructure.
    \item Chapter~\ref{ch:LQI}: We present a methodology to measure the quality of life in a city based on the pedestrian accessibility to amenities and services. We apply the methodology to Budapest and show how it can be used to capture inequalities in neighborhoods. 
    \item Chapter~\ref{ch:Conclusion}: We review the main contributions from this thesis, and outline future streams of work and open questions. 
\end{itemize}