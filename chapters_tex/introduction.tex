\chapter{Introduction: cities and newtorks}
%Why study cities? 

\epigraph{Cities have the capability of providing something for everybody, only because, and only when, they are created by everybody.}{Jane Jacobs, \textit{The Death and Life of Great American Cities} (1961, p. 238)}

The cities' capability of providing something for everybody is not only limited to its inhabitants and their daily life. As a complex system, cities offer a fertile study ground from multiple perspectives. Urban studies, from sociological perspectives to more technical ones such as engineering, have engage in analyzing different layers of urban life. Cities, as a giant evolving being are full of complexities, not only on the dynamics of how we live in them, but also on the interplays between multiple infrastructures that sustain urban life. This complexity makes of cities the perfect interdisciplinary field of study.

\section{From architecture to network science, a personal journey}

As an architect my first approach to study cities was from the buildings perspective, understanding how, by building, we delimit and shape spaces that model experiences in the city~\cite{ghel1971life}. These building are fundamental to define public spaces, mobility infrastructures, and even services that enable us to inhabit the \textit{ville}~\cite{sennett2018building}. The way we shape the city has an influence in how we inhabit it. Thus, understanding its infrastructures is fundamental to understand and plan better cities for a complex future. Specially, when tackling urban mobility challenges, as the way we plan, build and use the mobility infrastructures is fundamentally entangled with the livability of our cities.

%Talk about curiosity, why study cities from the complex systems perspective, and specially from network science.
After working in designing public policies to promote bicycle infrastructure in my home-city's government, I became interested in getting to understand the relation between different transportation modes, and how cities and their mobility infrastructures could be studied using large scale data in a systematic way. This curiosity led me to the complex systems field, and the use of network science to study cities.

\luis{write more about bridgning architeture and network science.}
\begin{itemize}
    \item Public spaces
    \item Urban mobility
    \item Complexity, cities as fractals (Batty)
    \item The always evolving city. We cannot predict the future, we can build it. Tools to understand the present and build a sustainable future.
    \item Personal reflection of interdisciplinary approaches (social sciences benefit from quantitative methods, but it is important to acknowledge interdisciplinarity as a two way street, recognize the importance of social sciences, and the long tradition of urban studies. Not because we have a hammer, should we treat everything as if it were a nail) \luis{This idea might go to conclusions}
\end{itemize}
% While predicting how the future city will be is an impossible task, understanding how it has evolve and what is its state of development is possible. This understanding 


\section{Aim and structure of the thesis}

\luis{I still have to write about the thesis, from where does it start (complex systems and the study of cities) to the main contributions. I anticipate two paragraphs for the complex systems and cities, and two more paragraphs for the contributions/overview.}

The primary aim of this thesis is to contribute to the better understand of the structural properties of multimodal transportation networks in urban areas. For that, I build on the tools and methods of network science to analyze the underlying complexity of urban mobility infrastructure. 

In the first part of this thesis I focus the attention in the multimodal infrastructure networks, first providing a review of previous works, and then giving an original contribution on the analysis of multilayer transportation networks. In the second part of the thesis I focus the attention to specific layers, first the bicycle infrastructure one, and then the pedestrian.

This thesis is structured as follows:

\begin{itemize}
    \item Chapter~\ref{ch:litReview}: We provide a review of previous work on multimodal transportation and mobility research from the complex system perspective. First by focusing on the infrastructure, and measure to quantify it, then on the mobility dynamics on top of these infrastructures, and finally we offer a review of available datasets and tools to analyze this specific type of networks.
    \item Chapter~\ref{ch:OverlapCensus}: We make an original contribution to the field by proposing a method to extract the multimodal profile from a city multiplex transport network. We apply our methods to fifteen cities, finding clusters of cities with similar multimodal infrastructure.
    \item Chapter~\ref{ch:BikeGrowth}: In this chapter, we focus our attention in the bicycle layer of the multimodal network, and propose algorithmic approaches to improve its connectivity. We found that focalized investment has the potential to rapidly improves the connectedness and directness of the bicycle infrastructure.
    \item Chapter~\ref{ch:LQI}: We present a methodology to measure the quality of life in a city based on the pedestrian accessibility to amenities and services. We applied the methodology to Budapest and show how it can be used to capture inequalities in neighborhoods. 
    \item Chapter~\ref{ch:Conclusion}: We review the main contributions from this thesis, and outline future streams of work and open questions. 
\end{itemize}