\chapter{Introduction}

Corruption is a major cause of suffering around the world. It slows economic growth~\cite{mauro1995corruption}, stifles innovation~\cite{rodriguez2014quality}, and increases inequality~\cite{gupta2002does}. The social impacts of pervasive corruption in a society, reflected for example in distrust of government and strangers~\cite{rothstein2013corruption} and in the low quality in government services~\cite{lambsdorff2001corruption}, suggest that corruption reinforces itself in a kind of feedback loop. Feeling that society is rigged, individuals adapt to their circumstances and play by the local rules. Like many other social scientific problems, understanding the emergence or persistence of corruption is difficult because it manifests both in individuals and societies. 

This dichotomy is one reason why there are multiple definitions of corruption. The World Bank and Transparency International define corruption as ``the abuse of public or corporate office for private gain'' and ``the abuse of entrusted power for private gain,'' respectively~\cite{nye1967corruption,rose1978corruption,transparency2007global}. This definition provides a good benchmark to use when deciding whether an individual's behavior should be considered corrupt. Recently, scholars of government and institutions have focused on a more macro-oriented definition of good governance, namely: ``the degree to which the exercise of public authority follows the principle of universalism or impartiality''~\cite{rothstein2008quality}. The presence of partiality or particularism in the decision making of public actors is a useful definition of corruption because it defines a norm of behavior that can be applied to various contexts~\cite{mungiu2013controlling}.

These two perspectives have had some success in quantifying causes and effects of corruption, building measures of the prevalence of corruption in countries and regions, and proposing policy interventions to improve the control of corruption in society. They share several important core concepts, for example both reference corruption as a kind of behavior existing at the intersection of the public and private domains. Both refer to the exploitation of power or advantage, presumably to the detriment of a wider group of people.

The definition of corruption based on the lack of impartiality has become increasingly relevant as researchers of corruption focus on grand corruption, distinct from petty corruption. Corruption is deemed petty when it is impersonal and transactional - for example when someone bribes a driving instructor to pass an exam or a policeman to avoid a speeding ticket. Grand corruption refers to coordinated, organized behavior to siphon resources to specific groups or networks of people~\cite{Mungiu-pippidi2006,fazekas2016}. When examples of grand corruption are discovered they are typically front page headlines, as for example in the case of Petrobras, Brazil's state-owned petroleum company, whose executives were found to have taken nearly \$10 billion of bribes and kickbacks from a group of 16 large construction firms in exchange for awarding them overpriced contracts~\cite{watts2017operation,ribeiro2018dynamical}. Over 160 people were arrested and 93 convicted, including the former President of Brazil.

 Such a large conspiracy with such significant payoffs to its participants is only possible as a collective effort of many people. Despite having a larger target to aim at, it is difficult for authorities to combat grand corruption. This is because grand corruption is often organized in a sophisticated way, with its actors having specific responsibilities and roles, and connections organized in a way to limit their vulnerability as a whole~\cite{ferrali2018corruption}. Organized crime groups~\cite{calderoni2011strategic} and the September 11th hijackers~\cite{krebs2002mapping} structured themselves in a similar manner. Grand corruption is a difficult topic for academic research for these same reasons and also because it is unlikely that large conspiracies can be understood in terms of a sequence of ``abuses of public or corporate office for private gain''. Somehow such conspiracies are much more complex social outcomes than the sum of the individual behaviors of their members.
 
 Throughout this thesis when we speak about corruption, we will be referring to grand corruption and its organization, rather than instances of petty corruption. Though the prevalence of bribery and its acceptance in society are likely correlated with the prevalence of grand corruption, it is not essential to its functioning.

While framing corruption in terms of the norms and rules of society may be more applicable to the study of grand corruption than the transactional view of corruption focusing on bribery, it suggests an over-socialized description of corruption. In an extreme interpretation, the environment determines the actions of individuals, who merely ``internalize norms and seek [to conform] to the expectations of others''~\cite{wrong1961oversocialized} and so participate in corruption. Such a perspective not only fails to give a satisfying answer to the question of why certain places have developed a good control of corruption while others have not, but is also empirically unsupported. We will see that there is significant variation in the prevalence of corruption as we measure it within countries, even at the level of towns. It also has limited ability to explain how the prevalence of corruption in a place might change over time. Indeed, the Petrobras scandal offers some hope for Brazil: in the end powerful people were imprisoned for their corruption.

As they focus on individuals and societies, respectively, both definitions avoid mentioning that corruption, like nearly all socioeconomic activity, happens between actors. In this thesis we consider corruption as a networked phenomenon~\cite{granovetter1985economic}. It is a property of the interactions between groups of actors such as people, firms, institutions, which evolves as these interactions change. In his 1985 paper on the embeddedness of economic action in social structure, Mark Granovetter suggested that ``force and fraud are most efficiently pursued by teams, and the structure of these teams requires a level of internal trust--``honor among thieves''-- that usually follows preexisting lines of relationship.'' In other words, Granovetter is suggesting that particular social structures or networks are required to scale corruption to high levels. We do not discard the productive micro-level definitions of corruption by the World Bank or Mungiu-Pippidi, but we do change the way they are applied.


We will demonstrate that this approach can describe in novel ways the organization and roots of corruption at various scales from towns to nations. It complements micro and macro-level frames of corruption by considering what happens in between.

Theory aside, why might networks be a useful in the study of corruption, for example in its measurement, detection, or diagnosis? Grand corruption, as its name suggests, requires organization and coordination. Such organization may manifest explicitly as a social network of specific actors, say members of parliament and heads of firms. It should also leave fingerprints in the interactions between firms and institutions doing business and exchanging money. Crucially, several recent developments make it possible to examine empirically the relationship between corruption and the networks of actors. The growth of the internet and its use by governments has created a wealth of fine-grained administrative data on interactions between the public and private sectors. 

One distinguished example of such data is information on public contracting or procurement markets, which accounts for upwards of 20\% of GDP in OECD countries~\cite{oecdprocurement}. Recently, researchers have developed ways to measure corruption risk in the award of such contracts~\cite{fazekas2016corruption,fazekas2017uncovering}. This data and measurement approach provides us with the micro-level data to undertake a network-based analysis of the phenomenon of corruption. We argue that this data, at present, offers the best possible approach to an evidence-based, data-driven anti-corruption research program as proposed by Mungiu-Pippidi in 2017~\cite{mungiu2017time}.

Nearly in parallel, researchers studying biological~\cite{alm2003biological}, ecological~\cite{ings2009ecological}, social~\cite{borgatti2018analyzing}, economic~\cite{schweitzer2009economic,jackson2010social}, and spatial ~\cite{barthelemy2011spatial} phenomena have made fundamental advances in their fields by using a network perspective. A by-product of these minor revolutions in different physical and social sciences is the emergence of set of tried and tested methods for studying networks within data. These efforts extend the pioneering work of sociologists and anthropologists~\cite{moreno1941foundations,freeman1978centrality,wasserman1994social}, who have been working on social networks since the 1940s, to new contexts and larger scales\footnote{For an excellent history of research on social networks we refer the reader to the review of Freeman~\cite{freeman1996some}. For an overview of the current relationships between the different traditions of network analysis, we refer the reader to Hidalgo's review~\cite{hidalgo2016disconnected}.}. This recent work on networks forms an emerging field of research called network science, which has created a rich tool set for the study of complex systems.


This thesis proposes to leverage these developments to extend the state of the art in corruption research. Specifically, we claim that corruption risk can effectively be measured at the micro-level using administrative public contracting data, and that by using network science methods we can describe the organization or structure of corruption. We do this in three contexts: relating social networks and corruption outcomes of towns, describing the relationship between corruption risk and market structures at the scale of countries, and analyzing the emergence of illegal cartel behavior among firms. In all three chapters we provide both a theoretical contribution to the understanding of corruption and policy implications. 

First we describe how the structure of social networks are related with the prevalence of corruption in local government contracting using data from Hungarian settlements. By linking social structure with corruption we strengthen our claim that corruption is in general a networked phenomenon. 

Next we zoom out to the level European countries, mapping their procurement markets as bipartite networks of public institutions and the firms they contract with. These networks have rich structure related to their level of corruption risk. By observing how corruption risk is distributed in these networks and how actors with high risk scores respond to political shocks, we highlight novel distinctions between the organization of corruption in different countries. Among EU countries we find significant heterogeneities, for example that in some countries corruption risk is concentrated in the core of the procurement markets, while in others it is more common in the periphery. 

Finally, we shift focus to cartels in procurement markets, transferring the perspective we have developed to study corruption to a different but related economic problem. Cartels are groups of firms which illegally avoid competition to maximize their profits. We propose a framework to map markets of competing firms using contracting data, to identify groups of frequently interacting firms, and to measure their potential for forming cartels. 

The remainder of the thesis is structured as follows:
\begin{itemize}
\item Chapter 2: Here we review past work on the conceptualization and measurement of corruption, introducing measures based on both perceptions and administrative data and comparing them. We also review the literature on network aspects of corruption.
\item Chapter 3: In this chapter we relate the social networks of Hungarian towns using data from an online social media portal with the amount of corruption risk in their local governments.
\item Chapter 4: In this chapter we quantify corruption at the national level using data on public contracts awarded by member states of the European Union. By mapping procurement markets as networks, we can examine the distribution of corruption risk in different countries, and observe how they react to political shocks.
\item Chapter 5: In this chapter we apply network methods to the problem of cartels, transferring the principles developed in earlier chapters to a domain adjacent to corruption. We consider the emergence of illegal cooperation among firms in procurement markets using records of their bidding behavior. We observe a ``hot spots'' in the network topology where collusion appears much more sustainable.
\end{itemize}

We conclude by tying our findings together, presenting some of their diagnostic interpretations, and suggesting future avenues of research.