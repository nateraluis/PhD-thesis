\chapter{Conclusion and open questions}\label{ch:Conclusion}

This thesis set out to contribute to the characterization of multimodal transport infrastructures, develop network science data-driven methods to address the open research problem of planning and identifying strategies to improve sustainable mobility in cities. We leverage the availability of open high quality urban infrastructure data sets to build the multiplex transportation network of multiple cities, we analyze the networks first at their multiplex configuration, and later on focus our attention into the bicycle and pedestrian layers.

We started this thesis with a review on Chapter \ref{ch:litReview} of previous works on multimodal transportation and mobility research from a complex systems' perspective. 
%Talk more about what has been done and the gap to address.

We have seen that the multiplex structure of urban mobility systems has similarities between multiple cities around the world. The findings presented in Chapter \ref{ch:OverlapCensus} suggest that it is possible to identify and compare those similarities in a systematically and rigorously way using the ``overlap census'' method. These similarities produce clusters of cities with similar multimodal configurations, such as those that have lack investment in their sustainable mobility options thus having more car-centric profiles, and clusters of cities with a more balanced mobility options.

At the single layer level, we saw in Chapter \ref{ch:BikeGrowth} that a common characteristic of cities is the fragmentation of their bicycle infrastructure networks. We proposed the use of data-driven algorithms to consolidate those components into connected networks to efficiently improve sustainable transport. The two proposed algorithms, when compared with two baselines, highlight the usefulness of growing the bicycle network on a citywide scale (considering all areas of the city) rather than randomly adding bicycle infrastructure. The proposed approach is not the last word in this development, since it does not yet explicitly optimize for directness and does not account for transport flow. As pointed out, the use of data-driven algorithms to identify crucially missing links in bicycle infrastructure has the potential to improve the mobility infrastructure of cities efficiently and economically.

Finally, in Chapter \ref{ch:LQI} we presented a data-driven, network-based method to quantify the liveability of a city based on pedestrian accessibility to amenities and services. We applied the methodology to Budapest and showed that it is able to capture inequalities between neighborhoods and districts in the city. When comparing our findings to the average real state prices we found a positive correlation, the higher the quality of life, the higher the average real state prices.    Our framework demonstrates a way to leverage open data sources to evaluate the quality of life and pedestrian accessibility in systematic city-wide scale.

The three papers contribute... 
%Write contribution to the field(s)

\subsection*{Future Work}
What is next.
% The future for data-driven anti-corruption research looks bright. Data quality and access are generally improving. New sources of data will go a long way to addressing some of the shortcomings and limitations of the work in this thesis. For example, data on company owners and board members and the economic and social relations of politicians and regulators have potential to extend the scope of the work presented in this thesis. Network methods are clearly applicable in these contexts as well. For instance, by tracking social network connections of firm leaders to people in power, one could measure the extent of political corruption in a country by the effect of such connections on profitability.

% A major challenge in corruption research using big data will be to carry out causal inference. Though data collected at large scales has many advantages, for instance allowing us to observe whole markets across significant periods of time, it seems unrealistic to carry out experimental studies at the same scale, especially without the participation of government bodies. Further research is needed to extend methods of causal inference, for instance as often applied to panel data by economists, to the setting of networks.

% In the absence of causal identification, the scientific value of the methods developed in this thesis can be tested in other ways. If network analyses of procurement based risk indicators can predict corruption cases that authorities, researchers, or journalists can confirm, that would lend additional credibility to our approach. One can also strengthen the validity of these methods by finding evidence of other kinds of white-collar crime occurring among distinguished actors, following the classical adage that ``where there is smoke, there is fire''.

% The involvement of governments in anti-corruption research presents both opportunities and dangers. Public actors willing to experiment with rules or enforcement can help overcome problems of causal interpretation inherent in the study of observational data. Such collaborations also have the greatest potential for real-world impact, clearly. However, if researchers of corruption focus their attention too much on such collaborations, they would introduce a significant bias to our understanding of corruption. Indeed, where corruption is endemic, it is unlikely that relevant government bodies would be willing to participate in effective anti-corruption studies. Engagement with super-national actors with some independent authority in certain locations, for example with the World Bank in its procurement-based development projects or the EU and it cohesion funds, would overcome this issue to some extent. Certainly, the first step in this direction would be to convince individuals in these organizations of the value of network methods in the study of corruption. We hope that this thesis will be useful in this regard. 

