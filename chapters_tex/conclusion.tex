\chapter{Conclusion and open questions}\label{ch:Conclusion}

This thesis set out to contribute to the characterization of multimodal transport infrastructures, develop network science data-driven methods to address the open research problem of planning and identifying strategies to improve sustainable mobility in cities. We leveraged the availability of open high quality urban infrastructure data sets to build the multiplex transportation network of multiple cities. We analyzed the networks first at their multiplex configuration, and later on focused our attention on the bicycle and pedestrian layers.

We started this thesis with a review of previous works on multimodal transportation and mobility research from a complex systems' perspective. In Chapter \ref{ch:litReview}, we covered the science of the dynamics of mobility: How do people move, and which forms of transportation they use? On the other hand these dynamics take place on an underlying (infra)structure which can be modeled by multilayer networks. We offered an overview of mathematical metrics developed in network science recently to the study of the topic. 

After discussing the state-of-the-art research on multimodal mobility and transport infrastructure, we leveraged on the mathematical representation of multilayer transport networks to uncover similarities and differences between multiple cities around the world. The findings presented in Chapter \ref{ch:OverlapCensus} suggest that it is possible to identify and compare those similarities in a systematic and rigorous way using the ``overlap census'' method. These similarities produce clusters of cities with similar multimodal configurations, such as those that have a lack of investment in their sustainable mobility options thus having more car-centric profiles, and clusters of cities with more balanced mobility options.

At the single layer level, we saw in Chapter \ref{ch:BikeGrowth} that a common characteristic of cities is the fragmentation of their bicycle infrastructure networks. We proposed the use of data-driven algorithms to consolidate those components into connected networks to efficiently improve sustainable transport. The two proposed algorithms, when compared with two baselines, highlight the usefulness of growing the bicycle network on a citywide scale (considering all areas of the city) rather than randomly adding bicycle infrastructure. The proposed approach is not the last word in this development, since it does not yet explicitly optimize for directness and does not account for transport flow. As pointed out, the use of data-driven algorithms to identify crucially missing links in bicycle infrastructure has the potential to improve the mobility infrastructure of cities efficiently and economically.

Finally, in Chapter \ref{ch:LQI} we presented a data-driven, network-based method to quantify the liveability of a city based on pedestrian accessibility to amenities and services. We applied the methodology to Budapest and showed that it is able to capture inequalities between neighborhoods and districts in the city. When comparing our findings to average real estate prices we found a positive correlation: the higher the quality of life, the higher the average real estate prices. Our framework demonstrates a way to leverage open data sources to evaluate the quality of life and pedestrian accessibility in systematic city-wide scale.

The literature review (Chapter \ref{ch:litReview}) and the three original chapters (\ref{ch:OverlapCensus}, \ref{ch:BikeGrowth}, \ref{ch:LQI}) directly contribute to the study of multimodal mobility and multilayer transport networks from the network science field. Specifically the literature review is one of the first academic reviews dedicated to covering the topic of multimodal urban mobility and multilayer transport networks from a complexity science perspective.
 
Not only did this research contribute to the growing academic literature, but it also has direct practical application from data-driven policymaking and urban planning perspectives. For instance, the contribution of Chapter \ref{ch:OverlapCensus} on the use of the overlap census to capture similarities among multimodal urban transport networks unravels how different transport modes are interlaced, helping to identify which layer (or set of layers) could be improved to promote multimodal sustainable mobility.

The proposed algorithms and their results in Chapter \ref{ch:BikeGrowth} showed that it is possible to systematically and effectively improve the connectivity of bicycle infrastructure networks. The use of data-driven algorithms to identify crucially missing links in bicycle infrastructure has the potential to help transportation departments and decision makers to improve the mobility infrastructure of cities efficiently and economically. This approach is not only useful for planning urban infrastructures, but could also be used together with  mobility flows simulations to provide insights on how the system will behave after new measures are implemented.

At last, the results presented in Chapter \ref{ch:LQI} highlighted the use of open data and network-based tools to quantify life quality as a function of walkability on urban networks. Such methodology is able to capture urban accessibility inequalities in a detailed manner. A data-driven approach has the potential to help decision-makers tackle social and environmental challenges better. The use of open data sources and the development of algorithmic approaches adds up towards a systematic framework for understanding urban liveability.


\subsection*{Future Work}

The increasing availability of urban and mobility data, together with the continuous growth and developments of computational capabilities and methods are providing us with new research opportunities. For example, data on multimodal mobility, such as digital tickets or wi-fi signals in public transportation, can help to uncover the multimodal mobility dynamics in cities. Furthermore, data sources from share mobility services, as shared bicycles and vehicles, can help us integrate such services with traditional public and private transport infrastructures. Multimodal integration and frameworks are becoming necessary to ensure real-time and user-centered solutions for planning, forecasting and managing services, while increasing safety, and reducing congestion or emissions.

A major challenge for cities will be to integrate and efficiently expand their different multimodal transportation options. The use of data-driven algorithmic approaches for planning bicycle infrastructure networks is one step towards a systematic framework for realistic bicycle network growth strategies. These strategies and algorithmic approaches should consider  qualitative updates, such as integrating planning cultures and processes, along with improving the quantitative methods, for example with the creation of redundant paths that improve the directness and coverage of the bicycle infrastructure. Further research is needed to account for transport flow, and to improve the possibilities for multimodal transport, such as integration with public transport, and the creation of interchange infrastructures.

We have showcased that it is possible to measure the quality of life on urban environments as a function of pedestrian accessibility to amenities. The proposed approach is not the last word on the topic since it does not yet account for other variables, such as the quality of services and infrastructure, along with other variables. Interdisciplinary research efforts will be needed to integrate data-driven algorithmic approaches with qualitative variables. Doing so will let us understand not only the physical aspects that drive the quality of life in cities, but also the qualitative qualities and perceived life quality by their inhabitants.

Understanding cities and their underlying mobility infrastructures is of paramount importance for developing sustainable cities. Indeed, cities and their sustainable mobility infrastructures are one piece in the puzzle towards reversing the global societal threat of climate change. This thesis is our contribution to the understanding of cities, and the development of methods and tools to build a better urban future. 