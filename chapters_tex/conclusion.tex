\chapter{Conclusion and open questions}\label{ch:Conclusion}

Conclusion for chapters and thesis
\luis{Personal reflection of interdisciplinary approaches (social sciences benefit from quantitative methods, but it is important to acknowledge interdisciplinary as a two-way street, recognize the importance of social sciences, and the long tradition of urban studies. Not because we have a hammer, should we treat everything as if it were a nail)}
% This thesis set out to demonstrate the utility of a network perspective on corruption. Noting that previous approaches to the study of corruption are either often overly atomic or overly structural, we framed corruption as an emergent phenomenon occurring between actors within the networks of their interactions. We leveraged the recent proliferation of data on public procurement to create transaction-level measures of corruption risk, which we embed in social and economic contexts. Our approach is validated by the observation that bad behavior at the dyadic level (i.e. between a corruption issuer and winner or among colluding firms) is related to and to some extent predicted by the network the dyad is embedded in~\cite{granovetter1985economic}.

% We have seen that social networks in a place relate to the prevalence of corruption in its local government. The findings of Chapter 3 suggest that certain modes of social organization facilitate corrupt behavior. These modes are increasingly observable and measurable at the level of societies because of the vast amount of data created by the use of modern telecommunication services. This suggests why corruption is a stubborn phenomenon, and why most interventions don't work.

% At the national level, we saw in Chapter 4 that corruption risk is distributed in very different ways across EU countries, nearly independent of the overall level of corruption risk observed in a country. In some countries with high levels of corruption risk, corruption is significantly concentrated in the relationships between core issuers and winners, while in other countries corruption is more concentrated among peripheral actors. Corruption risk tends to cluster in all countries in our analysis, though the extent to which it does varies greatly. Finally, we also observe significant heterogeneity in the response of corrupt relationships to political shocks. These differences have significant implications for anti-corruption strategies, casting doubt on global solutions. Indeed one could say that while countries with little amounts of corruption are similar, each highly corrupt country is corrupt in its own way.

% Finally, we presented a method to detect potential hot-spots for the emergence of illegal collusion in competitive markets using data on bidding. Using two empirical cases and a simulated model we showed how groups occupying specific positions in the co-bidding network topology are uniquely able to sustain the cooperation needed to maintain a cartel. Our framework demonstrates a way to reduce the vast complexity of a market to highlight groups of firms worth investigating.  

% Given previous work on the networked nature of criminal conspiracies, it is perhaps no surprise that a study of corruption and collusion in procurement markets found significant relationships between network structure and bad behavior. We argue that the specific methods developed and analyses carried out offer actionable insights into how corruption works in different contexts. They also offer a kind of blueprint for future analyses, showing how procurement data can be analyzed using network methods.

\subsection*{Future Work}
What is next.
% The future for data-driven anti-corruption research looks bright. Data quality and access are generally improving. New sources of data will go a long way to addressing some of the shortcomings and limitations of the work in this thesis. For example, data on company owners and board members and the economic and social relations of politicians and regulators have potential to extend the scope of the work presented in this thesis. Network methods are clearly applicable in these contexts as well. For instance, by tracking social network connections of firm leaders to people in power, one could measure the extent of political corruption in a country by the effect of such connections on profitability.

% A major challenge in corruption research using big data will be to carry out causal inference. Though data collected at large scales has many advantages, for instance allowing us to observe whole markets across significant periods of time, it seems unrealistic to carry out experimental studies at the same scale, especially without the participation of government bodies. Further research is needed to extend methods of causal inference, for instance as often applied to panel data by economists, to the setting of networks.

% In the absence of causal identification, the scientific value of the methods developed in this thesis can be tested in other ways. If network analyses of procurement based risk indicators can predict corruption cases that authorities, researchers, or journalists can confirm, that would lend additional credibility to our approach. One can also strengthen the validity of these methods by finding evidence of other kinds of white-collar crime occurring among distinguished actors, following the classical adage that ``where there is smoke, there is fire''.

% The involvement of governments in anti-corruption research presents both opportunities and dangers. Public actors willing to experiment with rules or enforcement can help overcome problems of causal interpretation inherent in the study of observational data. Such collaborations also have the greatest potential for real-world impact, clearly. However, if researchers of corruption focus their attention too much on such collaborations, they would introduce a significant bias to our understanding of corruption. Indeed, where corruption is endemic, it is unlikely that relevant government bodies would be willing to participate in effective anti-corruption studies. Engagement with super-national actors with some independent authority in certain locations, for example with the World Bank in its procurement-based development projects or the EU and it cohesion funds, would overcome this issue to some extent. Certainly, the first step in this direction would be to convince individuals in these organizations of the value of network methods in the study of corruption. We hope that this thesis will be useful in this regard. 

