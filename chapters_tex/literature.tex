\chapter[Literature review]{Literature review: multimodal transportation and mobility in urban networks}\label{ch:litReview}

In this chapter we provide a comprehensive overview of complex systems approaches to multimodal transportation and mobility in urban networks. We cover multiple approaches that had been used to model cities using the multiplex framework, measures to quantify centralities and changes in the given networks and layers. After reviewing the mobility infrastructure and measures, we move our focus to the mobility dynamics on top of those layers, covering the study of urban mobility dynamics in public transport systems and at individual level. Finally, we conclude the chapter with an overview of the available data and computational tools for the study of urban mobility networks\footnote{A stand alone of this chapter is currently under review at Transportation. No preprint is yet available.}.
\pagebreak


In this chapter, we survey the literature on urban multimodal mobility, and on urban transportation infrastructure as multilayer networks. Here, we focus on the primal approach to networks~\cite{porta2006primal}, where streets and mobility infrastructure constitute the network links, and intersections (bus stops, subway stations, etc.) constitute the nodes of the network.

The remainder of this chapter is arranged as follows. Section~\ref{sec:multilayernetworks} introduces the mathematical concept of multilayer networks and related theoretical research underlying network science approaches to the topic. In Section~\ref{sec:multimodalinfrastructures} we discuss research on urban transport infrastructures, including their theoretical modeling (see Subsection~\ref{sec:multimodalinfrastructures}) as multilayer networks and their empirical characterization (see Subsection~\ref{sec:measuresinfrastructure}). In Section~\ref{sec:multimodalmobility} we focus on mobility, flows and navigation across these multimodal systems, and implications for transportation choices. In Section ~\ref{sec:datatools} we cover the relevant open datasets and the main software tools which can be used to analyse multimodal transportation systems. We conclude with an outlook and a summary of open questions for the research community in Section~\ref{sec:conclusions}. 

\section{Multilayer networks: A framework for multimodality}\label{sec:multilayernetworks}

Over the last decades, networks have emerged as a versatile tool to understand, map and visualize the interconnected architecture of a wide range of complex systems~\cite{albert2002statistical,dorogovtsev2002evolution, newman2003structure, boccaletti2006complex}, in particular spatially-embedded ones~\cite{barthelemy2011spatial, barthelemy2018morphogenesis}. Formally, a network -- or graph -- $\mathcal G = (\mathcal N, \mathcal L)$ consists of a set of nodes $\mathcal N$, and a second set $\mathcal L$ of edges, describing connections among unordered pairs of elements of the first set. This information can be conveniently stored into an adjacency matrix ${A=a_{ij}}$, where $i=1, \dots, N$ are the nodes, and $a_{ij}=1$ if there is a link between nodes $i$ and $j$, $a_{ij}=0$ if there is no link between $i$ and $j$. In transportation systems~\cite{lin2013complex}, nodes can represent the stations of a network, and links direct connections between them. The adjacency matrix can also include weights $W=w_{ij}$, where $w_{ij}$ are positive real numbers, for instance describing how strongly connected two nodes are. For spatial systems, weights are often taken as the reciprocal of the distance between two nodes, or the time it takes to travel from one to another, i.e. $w_{ij}=1/d_{ij}$ or $w_{ij}=1/t_{ij}$.

More recently, network scientists have put effort into characterizing the structure of systems which are formed by different interconnected networks. Such interconnected structures are natural for transportation systems but also for social and biological networks. An example of these interconnections are the largest transportation hubs in worldwide cities, where stations are routinely served by bus, underground and railway infrastructures.

Indeed, most urban transportation systems systemically rely on the interplay between different means of transportation. These systems can be conveniently described by \textit{multiplex} or \textit{multilayer} networks. Here we introduce the \textit{vectorial} formalism for multilayer networks~\cite{boccaletti2014structure, battiston2014structural}, widely used in most papers on multimodal transportation. An alternative description can be provided by a more mathematically involved \textit{tensorial} framework~\cite{dedomenico2013mathematical, kivela2014multilayer}.  

In multilayer networks, links of different types, describing for instance a different mean of transportation, are embedded into different \textit{layers}. Each layer $\alpha$, $\alpha = 1, \ldots, M$, is described by an adjacency matrix $W^{[\alpha]} = \{w_{ij}^{[\alpha]}\}$. In a multimodal urban transportation network with three mobility infrastructure layers, $\alpha=1$ can represent the bus network, $\alpha=2$ the underground network, and $\alpha=3$ the urban railway network, for example. The full transportation system $\mathcal M$ can be described as $\mathcal W = \{ W^{[1]}, \ldots,  W^{[M]}\}$. Nodes $i=1, \dots, N$ are labeled in the same order in all networks. 

In the case of transportation networks, simply identifying nodes of different networks as the same station might not provide the most complete description of the multimodal network. Take for instance the largest stations in mega-cities, like King's Cross - St. Pancras in London, Grand Central Station in New York or Hongqiao transportation hub in Shanghai. All of these are identified by a unique location (node index) $i$ across the different transportation layers. Yet, switching from one mean of transportation to another within the same station might require a non-negligible fraction of time and effort, given the complexity and size of the overall infrastructure. 

For this reason, it is often relevant to complement the description of the \textit{intra-layer} connections present in the system, with \textit{inter-layer} links associated to the cost, spatial distance or time required to switch layers. Inter-layer links between layers $\alpha$ and $\beta$ at a node $i$ can be encoded through the inter-layer matrix $C_i=\{c_i^{[\alpha \beta]} \}$, and all such inter-layer connections can be stored in the vector $ C = {C_1, \ldots, C_N}$. In this case, the full multiplex structure of the system is described by taking into account both intra-layer and inter-layer connectivity, hence $\mathcal M = ( W,  C)$. Inter-layer links may be neglected for many measures focusing on diversity~\cite{battiston2014structural}, as well as correlations~\cite{nicosia2015measuring} across the layers of the systems, relevant to assess the different roles and geographical spanning of the different mean of transportation of a multilayer network. 

Multilayer networks are a natural framework for multimodal transport networks. Indeed, the concept of multilayer networks in the transportation engineering field goes back to 1970’s when Dafermos ~\cite{dafermos1972traffic} proposed a formulation for the traffic assignment problem for multiclass-user networks. Since then, the term ``multilayer'' has been interchangeably used with ``multiclass'', ``multimodal'', and ``multiuser''. Shortly after, Sheffi ~\cite{sheffi1978transportation} proposed the notion of ``hypernetwork'' that was later redefined as ``supernetwork'' ~\cite{sheffi1985urban} in which decision-making can be modelled as a route selection over a multilayer or multimodal network. In the network science field one of the pioneering works introducing the framework and concept of ``layered complex networks''~\cite{kurant2006layered} explicitly focuses on the case of transportation systems, where a first layer encodes the physical infrastructure of the system, and the second one describes the flows on such infrastructure. Other early works on the topic also dealt with interconnected systems at the worldwide level, focusing on different modes of transport such as the multiplex airline networks~\cite{cardillo2013emergence}.

Noticeably, multimodal infrastructures seem to possess exclusive characteristics different from other multilayer networks. For instance, when tools to assess the redundancy of the different layers are considered, transportation networks are often found to be irreducible~\cite{dedomenico2015structural}. Differently from many biological systems, where layers often duplicate information to guarantee the interconnected system a high level of robustness, the layers of a multiplex transportation systems are purposefully engineered to be different, in order to maximise efficiency~\cite{latora2001efficient}. As a byproduct of this feature, multimodal systems are also often highly fragile~\cite{buldyrev2010catastrophic}, and sensitive to disruptions or failures of a single infrastructure~\cite{dedomenico2014interconnected} (For a more detailed view on the topic see the early reviews~\cite{boccaletti2014structure, kivela2014multilayer} and textbook~\cite{bianconi2018multilayer} covering the field). Aleta et al.~\cite{aleta2019multilayer} provides a more recent eye-bird view of the field. A thorough review of the measures and models used to analyse such systems can be found in Battiston et al.~\cite{battiston2017new}, whereas de Domenico et al.~\cite{dedomenico2016physics} gave a theoretical overview of spreading and diffusive processes on such systems. In the following sections of this chapter, we focus on findings of more direct relevance to the research community working with multimodal transportation and urban mobility. The division of related research across these two themes is not meant to be rigid, but rather serves as an indication of the core topic treated in the different works. 


\section{Multimodal infrastructures}\label{sec:multimodalinfrastructures}

As cities grow and add different transportation modes, understanding the transportation infrastructure and its interconnected nature is crucial to capturing patterns of urban mobility. Since the 1950s, fields ranging from Architecture to Urbanism and Transport Planning have grown a large body of literature studying the structure of cities and their transport systems. With the growth of the Complex Systems and Network Science fields, new models have been developed to study the complexity behind urban systems, and specifically mobility infrastructure.

Multimodal urban infrastructures can be represented as multilayer networks, in which each layer $\alpha$ represents a mobility infrastructure (e.g. subway, light railway, bus service, pedestrian or bicycle infrastructure), the set of nodes $\mathcal{N}$ are locations (e.g. bus stops, intersections, subway stations), and the set of edges $\mathcal{L}$ in layer $\alpha$ are the infrastructure links between nodes in the same layer (e.g. subway lines, bus routes, bicycle lanes, see also Section \ref{sec:multilayernetworks}). Modeling infrastructures is of great importance for understanding how urban systems work, and for the design of new sustainable mobility options. 

In the following we describe recent findings related to how transportation layers are coupled and grow. First, we review the main complex systems models of urban transportation. Then, we describe empirical findings related to the transport infrastructures.

\subsection{Modeling urban infrastructure}\label{sec:modelinginsrastructure}

The design of efficient single-layer transportation networks is a classical problem in the domain of transport optimization. As such, it is widely studied and often formulated as a bilevel mathematical problem, where in the upper-level the traffic planner makes decisions regarding management of the system, and in the lower level users choose route, travel mode, origin and destination of their travel in response to the upper-level decision. The design of multimodal transportation network, instead, has been less studied ~\cite{farahani2013review}. In fact, multimodality brings new challenges to the transportation network design problem, including issues related to the integration of the street network with public transit and active transportation networks (walking and cycling networks) in which travellers can choose to take multiple modes for a single journey ~\cite{zhang2014design,huang2018multimodal}.

One of the first contributions to the modeling of multimodal urban infrastructures was provided by de Cea et al.~\cite{decea2005equilibrium} who pointed out that most of the models from the transport community~\cite{boyce1994introducing,boyce2002sequential,decea2007transit} used to plan and simulate the effects of new transportation options and failed to consider congestion associated with transit modes. This shortcoming is particularly relevant when the models are used for infrastructure development and predict transportation equilibria in future years. 

The model proposed by de Cea et al.~\cite{decea2005equilibrium} takes the road network as the base layer. There, links have an average operating cost that takes into account different mobility options (e.g. cars, taxi, etc.). For every public transportation mode a new layer is defined, with their unique nodes and links. For these layers, the cost function of the public transportation links depends on the combined effect of travel, waiting, and transfer time. The model considers the existence of combined trips (e.g. car/metro, bus/metro, etc) and looks for an equilibrium condition under the assumption that every user chooses her route to minimize their average operation cost (Wardrop’s first principle). This means that at equilibrium, only non-congested routes have a minimum cost, while those without flow represent a more costly option. Travel time might be affected by the interplay of different transportation means. For instance, vehicle flow over the road network may induce longer travel times in the bus network. Similarly, a passenger might decide not to take the subway if it is too crowded. ~\cite{decea2005equilibrium} show that their model is able to find the equilibrium in the trips between origin and destination in a toy model, and can be successfully applied to real-world scenarios. For instance, a rich version of this model (which consisted of 13 user classes, 11 transport modes, and 450 zones) successfully informed the planning of the new metro line 5 in Santiago (Chile).

A similar philosophy was deployed by Li et al.~\cite{li2007parkride} who considered a transportation infrastructure of cars, combined walk-metro paths, and park-and-ride. Differently from the previous work in which the model take into consideration the availability of routes, here parking availability and time spent by car commuters while looking for parking, which can be considerable~\cite{shoup2017high}, is considered explicitly. The focus here is on the interplay and impact of park-and-ride (P\&R) schemes to encourage users to switch from car travel to subway and public transportation options when traveling to the cities' central area. The model proposed by Li et al.~\cite{li2007parkride} considers the effects of traffic conditions on travel demand, and incorporates elastic demand into the model to capture commuters’ responses to traffic congestion and availability of parking supply. The responses of a user include the decision to switch to another transportation option, or to not make the trip at all. For the public transportation layer, the model also takes in consideration discomfort that may result from crowded subways. Through numerical simulations, Li et al.~\cite{li2007parkride} found that it is possible to reach an equilibrium control which prevents the emergence of traffic jams in the city center by implementing a suitable P\&R scheme, which looks at the combined effect of cost at the P\&R sites, parking availability in the city center, as well as metro fares and frequency.

While in the above works the interplay of different transportation modes was introduced, they were not yet  modeled as a multilayer network. One of the first applications of this new modeling framework was presented by Morris et al.~\cite{morris2012transport} who developed a toy model to couple different transport modes, according to the following rules. First, \textit{N} nodes are placed at random within the unit circle, mimicking a spatial configuration typical of many cities, and are connected by a Delaunay triangulation. Second, to simulate another transport mode, a second layer is generated by drawing a subset of the previously generated nodes and a second Delaunay triangulation. Finally, the two layers are coupled with interlayer links when a node is present in both layers. Using this toy model and a simulated Origin-Destination matrix, Morris et al.~\cite{morris2012transport} investigated how fragile the network is to changes in supply and demand. They found that increasing travel speed in one layer tends to concentrate trips in the fastest layer, and also produces congestion in the nodes where is possible to change transport mode.

Similar results have been reported for coupled random networks~\cite{gao2017comprehensive}, as well as scale-free networks~\cite{zhuo2011traffic} where a similar modeling approach is used to mimic a real-world transport scenario~\cite{du2016physics}. In their work, Du et al.~\cite{du2016physics} used a two-layer traffic model, where one layer provides higher transport speed than the other, and applied a Particle Swarm Optimisation algorithm to optimize the transport system capacity and reduce congestion both in synthetic and real-world multimodal networks. A more detailed coverage of the findings can be found when discussing betweenness centrality and interdependence in Section~\ref{sec:measuresinfrastructure}. 

Gil~\cite{gil2014configuration} proposed to use open data from Open Street Map to model the multimodal infrastructure network of a given city as a combination of three layers. The first layer is the street network, where the nodes are intersections and links are streets. This layer, accounting for private transportation, was again the reference layer with respect to the other transportation modes in the system, i.e.~all other layers have to be connected to and interact with it. The second layer is the public transport layer that represents the stations as nodes. It links the stations whenever there is a public transport service between two stations. This layer is coupled to the street layer by the stations and their closest street intersection. Gil~\cite{gil2014configuration} also include land use in their model to measure urban accessibility. This framework was applied to the analysis of the Randstad city-region in the Netherlands. The model was tested under different parameters and layer combinations, measuring reachability of the different city areas through closeness and betweenness centralities. Comparison with ground truth data showed that betweeness centrality in the public transportation layer can be a good indicator of passenger flows.

Aleta et al.~\cite{Aleta2017Multilayer} exploited one step further the richness of multilayer networks for transportation systems, highlighting two possible frameworks. In the first framework, each bus or metro line on its own could be considered an independent layer. This approach is useful to have a realistic model of human mobility which takes into consideration transfer times and synchronization between single trips. Yet, it does not allow to evaluate the importance of an entire transportation mode. This issue is resolved in the second framework. Here all the lines of the same mode are combined into a so-called \textit{superlayer}, which is fundamental to study the interdependence and resilience of the whole system. 

Using the previously described frameworks, Aleta et al.~\cite{Aleta2017Multilayer} investigated the public transport systems of nine European cities. Following the first framework, the authors focused on some structural features of the emerging system such as the overlapping degree (sum of the node's degree in all layers~\cite{battiston2014structural}). They found that public transport infrastructures have some universal properties, and that the maximum overlapping degree is similar in all the systems, even if the number of layers is different. This depends on the fact that networks are embedded in a physical space, hence imposing some bounds on the maximum number of links of each node structural constraints. Following the second framework, Aleta et al.~\cite{Aleta2017Multilayer} investigated in detail the superlayers and found that -- suprisingly -- the nodes with the highest overlapping degree are not necessarily the ones with the highest superlayer activity~\cite{nicosia2015measuring} Indeed, some transportation modes (and in particular the bus layer) have a tendency for hubs which might be disconnected to the other transportation modes, leading to high overlapping degree but low multilayer activity. The prevalence of such hubs is relevant when considering the robustness of the whole system. As the study highlights, it is often easier to move a bus stop to a street nearby, even if it is a local hub where multiple lines stop, than solving a disruption in a subway station. Aleta et al.~\cite{Aleta2017Multilayer} also assess the importance of the superlayers based on the number of shortest paths that make use of the superlayer, a measure that we characterize later in Section~\ref{sec:measuresinfrastructure}.

It is important to note that cities and their transportation systems are not static in time. This means new transport modes may be introduced or extended, such as when new bus/subway stops are added. Recently, to plan for Open Streets during the COVID-19 pandemic, Rhoads et al.~\cite{rhoads2020planning} proposed an investigation of the relation between streets and sidewalks. First, the authors used percolation theory to examine whether the sidewalk infrastructure in cities can withstand the tight pandemic social distancing. They then proposed an algorithm that takes into consideration both the sidewalk and street layers while improving the sidewalk connectivity. Despite spatial constraints, Rhoads et al.~\cite{rhoads2020planning} showed that it is possible to widen the sidewalks and improve the pedestrian connectivity with a minimum loss in the road network. 

More in general, some works have focused on the growth and evolution of multilayer networks~\cite{nicosia2013growing,kim2013coevolution,nicosia2014nonlinear}, generalising preferential attachment mechanisms in different ways~\cite{barabasi1999emergence}. However, these works do not take into account spatial constraints, and are not well suited to describing the evolution of spatial networks. For this reason, there is ample potential for future work to develop growth models for multimodal infrastructures, for instance considering densification and exploration, as previously done for street networks~\cite{strano2012evolution}. An alternative view can be obtained by investigating the optimal growth and design of the multiplex structure of the different layers, as was done for the multilayer airline network composed by routes of different airline companies~\cite{santoro2018pareto}. 

The models discussed in this section have shown how the multilayer networks framework has been used to investigate the structure, function and vulnerabilities of a complex transportation system. This modeling framework will be used along this thesis, first in Chapter \ref{ch:OverlapCensus}, where we provide insights into the similarities and differences between the multimodal transport networks of multiple cities. Later on in Chapters \ref{ch:BikeGrowth} and \ref{ch:LQI} we will focus on the particularities of two layers, the bicycle infrastructure and the pedestrian one. In the next section we cover some measures to quantify the multiplexity of these structures and their interconnections.  

\subsection{Characterizing multimodal infrastructure}\label{sec:measuresinfrastructure}

The empirical study of urban transportation infrastructure has revealed some structural properties of specific transport networks~\cite{barthelemy2011spatial}. However, the question remains how to quantify the effectiveness of their interconnections? In this section, we describe a number of measures that have been used to capture the multiplexity of multimodal infrastructures, such as the importance of different nodes, the system's resilience, and the similarity between layers.

\paragraph*{Paths} 
At a more global scale, multimodality is often associated with the ability of an agent to navigate the system by using the available transportation modes. Journey or route planning in a multimodal network refers to finding the shortest route in a multilayer network when multiple transportation modes such as private car, public transit, walking, and cycling are combined in a single and integrated journey~\cite{zografos2008algorithms,botea2013multi}. For this reason, the navigation of an agent in a transportation network can be measured through the available types of paths.

The literature on multimodal route planning has been mostly built around static multilayer networks with no time-dependent characteristics~\cite{bast2016route}. Thus, a first possibility is to consider the quickest path, neglecting transfer and waiting times between transportation modes. This path is computed using the fastest speed associated to the edges, assuming a perfect synchronization between the different transportation modes. This is analogous to finding the shortest path between $i$ and $j$ in an aggregated weighted-single layer network, where the weight describing the time to travel between two nodes is the minimum among those offered by the different transportation options.

Yet, this approach often falls short in capturing real patterns of mobility. Indeed, transportation networks also have a temporal dimension that has to be taken into consideration. In order to find a path that allows for a change between two or more different transportation options we must find a time-respecting path~\cite{Gallotti2014Efficiency}, defined as the shortest path between nodes $i$ and $j$ that considers the departures and arrivals constraints given by timetables. In order to find such time-respecting paths, Huang et al. ~\cite{huang2019route} proposed a time-expanded model to account for the dynamic nature of route planning in a multimodal network with fixed public transit and dynamic ridesharing vehicles. However, a comprehensive route planning algorithm that simultaneously considers multiple transportation modes in a truly time-dependent network is missing. Furthermore, for multimodal transportation networks, the walking transfer time between modes has to be taken into account when computing time-respecting paths. 

Besides time constraints, in order to find a viable path between $i$ and $j$, the logic of the proposed sequence of modes has to be assessed~\cite{battista1996path,lozano2001path}. For instance, in some cases, a path composed by subway-bus-subway-car-subway might solve the shortest path, but the presence of private transportation (car) as an intermediate option makes it an ``illogical path'' and thus an unlikely choice for a user. The validity of transportation sequences can in general be formalized in terms of cost associated to change of transportation mode.

Finding viable paths is one of the most important problems in urban transportation, as it has the potential to help users finding the most efficient paths in the city. Lozano et al.~\cite{lozano2001path} have proposed an efficient algorithm to find such paths when the agent establishes her limitations on the number of modal transfers.

The contribution of the different layers of a multiplex networks to shortest paths might be very unequal~\cite{Aleta2017Multilayer}. For instance, the rail systems (trams, subway) contribute to most of the shortest paths in a city, connecting distant points at a greater velocity and in straighter routes than bus or other transport modes. However, such layers have only few stations. Slower and more local transport modes often serve a complementary role, offering a deeper coverage of the city.

\paragraph*{Spatial outreach}
The availability of different transportation modes, such as subways or tramways affects how easily it is to reach certain locations in the city. A way to measure this effect is to quantify the associated \textit{spatial outreach}~\cite{strano2015features}. The spatial outreach can be computed as the average distance from node $i$ to all other nodes in the same layer $\alpha$ that are reachable within a given travel cost $\tau$. Mathematically it is defined as follows:

\begin{equation}
    L_\tau(i)=\frac{1}{N(\tau)}\sum_{j|\tau_{m}(i,j)<\tau}d_e(i,j),
    \label{eq:outreach}
\end{equation}

where $d_e(i,j)$ is the distance between nodes $i$ and $j$, and $N(\tau)$ is the number of nodes reachable on the multilayer network within a travel cost $\tau$.

Strano et al.~\cite{strano2015features} modified the average speed (traversal time of a link) in the layers of multiplex systems to measure their effects on the corresponding travel outreach. Strano et al.~\cite{strano2015features} found that when the metro speed increases compared to the street layer speed, a clear area of high-outreach nodes emerges in the city center and around the nodes that have connections to the high-speed layer. In other words, as the velocity in layer $\beta$ increases, the nodes that are closer to the interchange nodes in layer $\alpha$ improve their accessibility, implying that a person can efficiently travel from this area to faraway places. This concept of travel outreach is similar to that of isochrones which quantify the accessible area from a given point within a certain time threshold, e.g.: What is the area that a user can reach traveling $x$ minutes, in any direction, from a given point? Biazzo et al.~\cite{biazzo2019accesibility} used this approach to measure accessibility in different urban areas computing the isochrones as a combination of public transit and pedestrian infrastructure. With this method, scores were obtained that capture how well a city is served by the public transit and how accessible a specific area is to the rest of the city.


\paragraph*{Betweenness centrality and interdependence}
The relevance of the nodes in a network is characterized by centrality scores. Additional to single layer networks, in multiplex networks the overall centrality of a location or station also depends on the interplay of the different transportation options.

The simplest way to define the centrality of a node is to measure its degree, or the number of locations directly connected to it. Such a measure, however, is not so relevant for systems embedded in space due to spatial constraints. For example, the number of possible connections of a given node (intersection) in the street layer is highly constrained by the physical space, as a single intersection can only have a limited number of intersecting streets/sidewalks. For this reason, other centralities than degree are typically used to assess the relevance of a location. One of such measures is the betweenness centrality~\cite{Freeman1977Centrality}, measuring the number of shortest paths passing via a given node. This measure is also called ``load'' and can be seen as the simplest proxy for traffic flow in the system as it assumes uniform demand between each pair of nodes. In the absence of explicit mobility data, betweenness centrality can be used as a proxy to assess the areas at risk to become overcrowded and to identify potential bottlenecks in the system. 

In multiplex networks, shortest paths from one node to another can to pass through links in two or more layers. This effect can be quantified by measuring the interdependence of a given node $i$ as:

\begin{equation}
    \lambda_i=\frac{1}{N-1}\sum_{j\neq i}\frac{\psi_{ij}}{\sigma_{ij}},
    \label{eq:coupling}
\end{equation}

where $\psi_{ij}$ is the number of shortest paths between $i$ and $j$ that use edges in two or more layers, and $\sigma_{ij}$ is the total number of shortest paths between $i$ and $j$~\cite{morris2012transport,battiston2014structural,strano2015features}. Node interdependence takes values in $[0, 1]$, with values close to 1 associated to a high coupling of the layers, while values close to $0$ mean that most of the paths from that node to other nodes go through just one layer. By taking the average over all nodes $\lambda = 1/N \sum_i\lambda_i$ it is possible to obtain a single score for the whole system. 

This interdependence measure can be modified to obtain a score for a specific layer~\cite{Aleta2017Multilayer}. Then the layer interdependence for layer $\alpha$ is defined as:

\begin{equation}\label{eq:layer_interdependency}
    \lambda^{\alpha}=\frac{\sum_i\sum_{i\neq j}\psi_{ij}^{\alpha}}{\sum_i\sum_{i\neq j}\psi_{ij}},
\end{equation}

where $\psi_{ij}^\alpha$ describes the number of shortest paths between nodes $i$ and $j$ using two or more layers and such that at least one of them corresponds to layer $\alpha$.

When applying this measure to multimodal transport networks, Aleta et al.~\cite{Aleta2017Multilayer} found that the metro and tram layers play an important role in concentrating shortest paths. For Madrid, Aleta et al.~\cite{Aleta2017Multilayer} found that more than 40\% of the trips have at least one link in the metro layer, even if the metro layer has only 241 nodes while the bus layer has 4590 nodes. 

As cities grow and new lines and transport modes are added into the mobility system, new interconnections between layers appear, changing the betweenness of the different nodes. Ding et al.~\cite{ding2018traffic} studied how centralities evolved when the rail network of Kuala Lumpur grew from a tree-like structure to a more complex one. Their findings suggest that, as the network grows, the average shortest path in the multilayer network can decrease dramatically, especially as new nodes are able to serve as interchange between layers, thus enabling new shortest paths along the system.

The results from Ding et al.~\cite{ding2018traffic} are in line with the previous findings by Strano et al.~\cite{strano2015features}, on how the subway layer affect the distribution of nodes centrality in London and New York. Strano et al.~\cite{strano2015features} show that the introduction of new interconnected layers affects the congestion of the street layer. In fact, the presence of the subway layer allows to move traffic from internal routes and bridges to the terminal points of the subway system which might be used as interchange locations for suburban flows into the city center. Theoretical work by Sol\'{e}-Ribalta et al.~\cite{sole-ribalta2016congestion} confirms that one of the main drivers affecting traffic dynamics and congestion in multimodal transport networks is the interchange from the least to the most efficient layers.

\paragraph*{Resilience}
Evaluating the robustness of a transport system under failures is an important task with practical implications in urban planning. Notably, multimodality affects significantly the resilience of transportation system~\cite{dedomenico2014interconnected}.

In a single layer network, the disruption of an infrastructure, i.e.~the removal of a link, can make a station or a part of the city disconnected. For example, for a transit station in a single layer network: If the links connecting the station with the rest of the system are removed, the station is inaccessible. However, if such a station is part of a multimodal transportation network, it could still be accessed through other layers. To measure the impact of multimodality on resilience, de Domenico et al.~\cite{dedomenico2014interconnected} used random walks (see Sec.\ref{mobility_1}) to mimic trips among locations and investigated the coverage time in the London's transportation system under different scenarios. De Domenico et al. showed that the interconnected nature of the different transport modes dramatically enhances the overall system resilience to failure compared with the single layers. A similar approach was followed by Baggag et al.~\cite{baggag2018resilience}, where the coverage time of random walks was used to measure the robustness of the multimodal transportation networks of Paris, London, New York, and Chicago. To mimic realistic trips, Baggag et al. introduced several constrains on the complexity of the trips, for instance limiting the maximum number of transport mode changes. More recently ~\cite{ferretti2019resilience} used the multiplex framework to model Singapore's public transportation infrastructure and test its resilience against floods in the city in different scenarios, finding that the system is extremely resilient as it faces the first significant disruption only after the removal of $~50\%$ of it edges.

\paragraph*{Overlap census}\label{overlap}
Urban transportation networks present different degrees of multimodality and integration. The overlap census proposed by Natera et al.~\cite{natera2020multimodal} proposed a method to observe, quantify and compare those differences. In Chapter~\ref{ch:OverlapCensus} we cover this original contribution in detail.

The overlap census was not the first measure to compare similarities between multiplex networks and layers. Indeed, similar approaches were developed for multiplex networks more in general. For a detailed view see Nicosia et al.~\cite{nicosia2015measuring} who proposed measures capturing nontrivial correlations in multiplex networks and models to reproduce those correlations. More recently, Brodka et al.~\cite{brodka2017similarity} also presented an overview of different metrics to compute similarities between layers in multiplex networks. 


\section{Multimodal mobility}\label{sec:multimodalmobility}

Understanding urban travel is paramount for a range of real-world applications, including planning transportation~\cite{patriksson2015traffic} and designing urban spaces. Starting from the 1950s, a large body of literature in the fields of Geography and Transportation has studied how people move and use transportation technology. 

As the transport infrastructure becomes increasingly multimodal, modeling how individuals make travel decisions in complex interconnected networks is critical. In recent years, the scientific understanding of human mobility has dramatically improved, also due to the widespread diffusion of mobile-phone devices and other positioning technologies, which allowed to gather large-scale geo-localized datasets of human movements and develop increasingly realistic behavioural models. Concurrently, these recent developments had benefited by the dramatic growth of the fields of Complex Systems and Network Science, which brought together ideal tools to study interconnected systems\footnote{For a comprehensive review of the recent literature stream of Human Mobility see~\cite{barbosa2018human}}. This new data-driven modeling framework for multimodal mobility was pioneered by Kurant et al.~\cite{kurant2006extraction}, who extracted data from public transportation timetables to characterize the mobility structure and traffic flow of a transportation network. Despite recent advancements, our understanding of multimodal mobility in urban systems remains limited, also due to the difficulties related to collecting comprehensive data across multiple transportation modalities. 

In this section we review the scientific literature on multimodal mobility. While our focus will be on multimodality, we will inevitably touch upon some concepts related more broadly to modeling of urban travel. In Section~\ref{mobility_1}, we briefly summarize existing models, focusing on latest advances driven by the Complex Systems literature. In Section~\ref{mobility_2}, we review measures and empirical findings, with a focus on recent studies based on passively collected data sources. 

\subsection{Modeling urban mobility \label{mobility_1}}

Modeling travel in a multimodal system involves understanding how individuals make decisions in a constantly changing complex environment. The most common family of models for travel demand in the Geography and Transportation literature are the \emph{four-step models} proposing that each trip results from four decisions~\cite{mcnally2000four}: 1) whether to make a trip or not, 2) where to go, 3) which mode to use, 4) and which path to take. For simplicity, these steps have been largely considered as independent, sequential choices, and correspond to four modeling steps: trip generation, trip distribution, mode choice, and route assignment\footnote{See the review of McNally~\cite{mcnally2000four} for a comprehensive overview about this modeling approach}.

In recent years, the field of Complex Systems has modeled travel behavior on multiplex networks using different approaches that we briefly review in this section. Complex Systems research has proposed novel individual~\cite{song2010modelling,jiang2016timegeo,alessandretti2020scales} and collective~\cite{simini2012universal,schlapfer2020hidden} models that capture well the first two aspects of travel behaviour: trip generation and trip distribution. In this review, we will focus largely on the last two of the four modeling steps, mode choice and route assignment, because they are the most relevant in the framework of multimodality. It is important to remark that, also due to the lack of empirical data on the mechanisms driving human navigation, many models rely on simplistic assumptions, for example that individuals are rational, homogeneous, or have unlimited knowledge. Research based on novel data sources will be key to developing mobility models on multilayer networks that include realistic elements such as limited knowledge and cognitive limitations.


\paragraph{Random walks.}
The random walk is one of the most fundamental dynamic processes~\cite{sole2016random} that has been widely studied in the Complex Systems literature as a prototypical model for numerous phenomena occurring upon networks, including human mobility. Importantly, in contrast to widely used models that assume individuals with global knowledge of the system thus choosing shortest routes~\cite{wardrop1952road}, random walks assume that agents are only aware of the local connectivity at each node. A random walk on a graph is defined by a walker that, located on a given node $i$ at time $t$, hops to a random nearest neighbor node $j$ at time $t + 1$. In the case of multilayer networks, the walk between nodes and layers can be described with four transition rules accounting for all possibilities~\cite{dedomenico2014interconnected}: (i) $P_{ii}^{\alpha\alpha}$, the probability for staying in the same node $i$ and layer $\alpha$; (ii) $P_{ij}^{\alpha\alpha}$ the probability of moving from node $i$ to $j$ in the same layer $\alpha$; (iii) $P_{ii}^{\alpha\beta}$ the probability of staying in the same node $i$ while changing to layer $\beta$; (iv) $P_{ij}^{\alpha\beta}$ the probability of moving from node $i$ to $j$ and from layer $\alpha$ to $\beta$, in the same time step. These probabilities depend on the strength of the links between nodes and layers, e.g. the frequency of vehicles and the cost associated to switching layers. 

Despite their simple formulation, random walks provide fundamental insights to many types of diffusion processes on networks and allow to measure a network's dynamical functionality. For example, random walk processes were used to measure the navigability of multiplex networks~\cite{dedomenico2014interconnected}. To this end, one can measure the coverage of the multiplex network $\rho(t)$, defined as the average fraction of distinct nodes visited by a random walker in a time shorter than $t$ (assuming that walks started from any other node in the network), and describing the efficiency of a random walk in the network exploration:

\begin{equation}\label{coverage}
    \rho(t)=1-\\\frac{1}{N^2}\sum_{i,j=1}^{N}\delta_{i,j}(0)\text{exp}[-\mathbf{P}_j(0)\mathbb{P}\mathbf{E}_i^{\dagger}],
\end{equation}

where $\mathbf{P}_j(0)$ is the supravector of probabilities at time $t=0$, the matrix $\mathbb{P}$ accounts for the probability to reach each node through any path of length $1, 2, \dots, \text{or}\ t+1$, and $\mathbf{E}_i^{\dagger}$ is a supra-canonical vector allowing to compact the notation. De Domenico et al.~\cite{dedomenico2014interconnected}, showed that the ability to explore a multilayer network is influenced by different factors, including the topological structure of each layer and the strength of interlayer connections and the exploration strategy. Further, they showed that the multilayer system is more resilient to random failures than its individual layers separately because interconnected networks introduce additional paths from apparently isolated parts of single layers, and thus enhance the resilience to random failures.

Random walks have further been used to assign a measure of importance to each node in each layer, by measuring the asymptotic probability of finding a random walker at a particular node-layer as time goes to infinity, the so-called \emph{occupation centrality}~\cite{sole2016random}. Sol\'{e}-Ribalta et al.~\cite{sole2016random} provided analytical expressions for the occupation centrality in the case of multilayer networks.

\paragraph{Travel time minimization approaches.}
At the other end of the spectrum, agents are assumed to have global (or nearly global) knowledge of the system. This is one of the most widely-used approaches in the transportation literature, rooted in Wardop’s user equilibrium principle~\cite{wardrop1952road} for traffic assignment. Under the user equilibrium principle, in a congested system, all agents choose the best route, e.g. no user may lower his transportation cost through unilateral action. Uchida et al.~\cite{uchida2005study}; Zhou et al.~\cite{zhou2008dynamic}; Verbas et al.~\cite{verbas2015dynamic} are among many of the studies that proposed formulations and solution algorithms for traffic assignment under user equilibrium in a multimodal system.

Complex Systems research has developed models where agents aim at minimizing their individual travel times in congested ~\cite{tan2014congestion,bassolas2020scaling,manfredi2018congestion,sole-ribalta2016congestion} or uncongested ~\cite{du2014traffic,du2016physics} networks. 

Bassolas et al.~\cite{bassolas2020scaling} developed an agent-based models describing mobility of individuals through a multilayer transportation system with limited capacity. The routing protocol used by individuals for planning is adaptive with local information. In the absence of congestion, individuals follow the temporal optimal path of the static multilayer network calculated by the Dijkstra algorithm. If there are line changes, Bassolas et al.~\cite{bassolas2020scaling} estimate besides the change walking penalty an additional waiting time of half the new line period (the real waiting time will be given by the vehicles' location in the line when the individual arrives at the stop). An individual's route is only recalculated when a congested node, whose queue is larger than the vehicle’s capacity, is reached. The work investigates analytically (for simple networks) and via numeric simulations the robustness of the network to exceptional events which give rise to congestion, such as demonstration concerts or sport events. The study revealed that the delay suffered by travellers as a function of the number of individuals participating in a large-scale event obeys scaling relations. The exponents describing these relations can be directly connected to the number and line types crossing close to the event location. The study suggested a viable way to identify the weakest and strongest locations in cities for organizing massive events.

Similarly, Manfreid et al.~\cite{manfredi2018congestion} introduced a limit to the nodes capacity of storing and processing the agents. This limitation triggers temporary faults in the system affecting the routing of agents that look for uncongested paths.

Importantly, the assumption that individuals have global knowledge of the system and minimize travel time contrast recent findings in spatial cognition, showing that human spatial knowledge and navigation ability is limited~\cite{gallotti2016limits,bongiorno2021vector}. For example, a recent study on pedestrian navigation made clear that path choices seem to be affected by the orientation of street segments along the route~\cite{bongiorno2021vector}. Recent modeling approaches for single-layer networks~\cite{manley2018exploring} incorporate these ideas in routing models where agents are characterized by bounded knowledge and limited rationality. Further research will be necessary to develop realistic multilayer routing models accounting for the limits of spatial cognition.


\subsection{Characterizing multi-modal mobility. \label{mobility_2}}

Traditionally, multimodal mobility models are calibrated using data from travel surveys: \emph{Revealed Preference} surveys retrieve actual travel information from the respondents, while \emph{Stated Preference} surveys expose the travelers to various hypothetical scenarios and record their choices~\cite{arentze2013travelers}. Studies based on survey data have provided insights into how multimodal travelers value aspects such as the different travel time components (in-vehicle time, walk time, access time, wait time...)~\cite{abrantes2011meta}, service quality~\cite{wardman2001review}, travel costs~\cite{arentze2013travelers}, and heterogeneities across socio-demographic groups~\cite{nobis2007multimodality}. Due to the high costs associated with data collection and inherent biases in self-reported data, these studies suffer from serious limitations, including small sample sizes, data inaccuracy and incompleteness~\cite{chen2016promises,zannat2019emerging}. Covering empirical results from travel surveys is outside the scope of this review, and we refer the reader to Arentze et al.~\cite{arentze2013travelers} for a comprehensive introduction to the topic.

In recent years, the empirical research on Human Mobility has taken new directions. A growing body of literature has focused on quantitative descriptions of human movements from large, automatically collected data sources, such as mobile phone records, travel cards and GPS traces~\cite{barbosa2018human}. In this section, we give an overview of recent empirical findings on multimodal mobility in the field of Complex Systems which focused on two important aspects: 1) the dynamics of public transport systems, whose study was driven by the availability of public transport data such as schedules and positions of stop and stations, and 2) individual multimodal behaviour driven by the availability of data collected using `smart travel cards' and GPS data. 

The existing empirical research on multimodal travel based on passively collected data-sources is far from being comprehensive. Most studies have focused on public transit, such that the interplay between public and forms of private transportation such as walking, driving and cycling has been poorly characterized. Further, several studies are based on public transport schedules instead of real-time data, thus neglect important effects deriving from congestion. The increasing availability of high-resolution GPS trajectories collected by individual mobile phones and sensors installed on private and public transport vehicles~\cite{barbosa2018human} will be key to filling these gaps in the literature.


\subsubsection{Public transport systems dynamics.} 
Over the last decade, the availability of detailed public transport schedules shared by public transport companies (see also Section~\ref{sec:datatools}) has allowed to better estimate travel times and characterize transport systems. 

\paragraph{Efficiency} 
To satisfy the demand of large number of individuals while reducing energy and costs, multilayer transport systems must achieve high efficiency. One aspect concerns the \emph{synchronization} between the network layers, because the more layers are synchronized, the less users have to wait for vehicles. The synchronization inefficiency $\delta(i,j)$~\cite{Gallotti2014Efficiency,barthelemy2016structure} for nodes $i$ and $j$ can be measured as the ratio of the time-respecting travel time $\tau_t(i,j)$, which accounts for walking and waiting times and the fact that the speed of vehicles varies during the day, and the minimal travel time $\tau_m(i,j)$, assuming that vehicles travel at their maximum speed and that transfers are instantaneous:

\begin{equation}
    \delta(i,j)=\frac{\tau_t(i,j)}{\tau_m(i,j)}-1
\end{equation}

Using the synchronization inefficiency, Gallotti et al.~~\cite{Gallotti2014Efficiency} showed that, on average in the UK, $23\%$ of travel time is lost in connections for trips with more than one mode. Interestingly, across several urban transport system in the UK, the synchronization efficiency $\delta(i,j)$ obeys the same scaling relation with the path length $\ell(i,j)$:

\begin{equation} \label{eq:deltaSynchronization}
    \delta(i,j) \approx \delta_{\textit{min}}+\frac{\delta_{\textit{max}}-\delta_{\textit{min}}}{\ell(i,j)^v},
\end{equation}

with $v \approx 0.5$ and where $\delta_{\textit{max}}$ and $\delta_{\textit{min}}$ are the maximum and minimum values of $\delta(i,j)$ for a given urban transport system.

Further, Gallotti et al.~\cite{Gallotti2014Efficiency} have shown that the average synchronization inefficiency $\overline{\delta}$ for a given urban system follows: 

\begin{equation}
   \delta \sim \Omega^{-\mu},
\end{equation}

where $\Omega$ is the total number of stop-events per hour (e.g. the number of times a vehicle stops), and $\mu = 0.3 \pm 0.1$.

Other studies focused on the efficiency in terms of ability to satisfy users demand. Alessandretti et al.~\cite{alessandretti2016user},introduced a method based on non-negative matrix factorization to compare the network of commuting flows and the public transport network. This methodology, applied to various public transport systems in France, showed that, while in Paris the transportation system meets the overall demands, it does not do so in smaller cities where people prefer to use a car despite having access to fast public transportation. Sui et al.~\cite{sui2019publictransport} proposed three topological metrics to quantify the interaction between public transport network and passenger flow and applied it to study differences between the cities of Chengdu and Qingdao in China. Hollecsek et al.~\cite{holleczek2014detecting} used data mining approaches to compare the use of public and private transportation and identify the existence of weak transportation connections.
 
\paragraph{Congestion.} 
Congestion can dramatically alter travel time estimates for routes that use popular network links in an urban system, but adding network layers in a multilayer transport system can help reduce global congestion~\cite{chodrow2016congestion}. Given a network with edges $e$ and flows $j_e$, one can quantify the total time lost in congestion as 

\begin{equation}
    T_c(\textbf{j})=\sum_{e\in \mathcal{L}}j_e(t_e^* - t_e(j_e)),
\end{equation}

where $\textbf{j}$ is the vector of flows whose $e$th element is the flow along edge $e$, $t_e^*$ is the free flow time on edge $e$ (in the absence of congestion), and $t_e(j_e)$ is the congested travel time in the presence of flow $j_e$. 
This measure can be used to analyze the impact of changes in flow along a route. The quantity

\begin{equation}
    \Delta_p = - \nabla T_c(\textbf{j}) \cdot \textbf{e}_p
\end{equation}

where $p$ is a path and $\textbf{e}_p$ is the vector whose $e$th component is 1 iff $e \in p$, quantifies the impact of removing a single unit of flow from $p$ on the global congestion function $T_c$. 
Chodrow et al.~\cite{chodrow2016congestion} quantified how the creation of a planned metro network in Riyadh would affect congestion, by quantifying the change $\Delta_p$ as a function of the speed ratio between the street and metro systems:

\begin{equation}
    \beta = v_{c}/v_{m}
\end{equation}

Using this approach, Chodrow et al. showed that, as the subway speed increases global congestion is reduced, but it increases locally close to key metro stations.

\paragraph{Navigability} 
As cities and their transportation systems become increasingly complex and multimodal, it is important to quantify our difficulty navigating in them. It has been shown that multilayer transport system are characterized by limited navigability, implying that finding one's way is cognitively challenging~\cite{gallotti2016limits}. To quantify the difficulty of navigating between two nodes $s$ and $t$ in a network, one can compute the total information value of knowing any of the shortest paths to reach $t$ from $s$:


\begin{equation}
S(s \rightarrow t) = - \log_2 \sum_{\{p(s,t)\}} P[p(s,t)]
\end{equation}

where  ${p(s, t)}$ is the set of shortest paths between $s$ and $t$ (note that there can be more than one with the same length) and $P[p(s,t)]$ is the probability to follow path $p(s,t)$, making the right choice at each intersection along the path~\cite{rosvall2005networks}:

\begin{equation}
    P[p(s,t)]=\frac{1}{k_s}\prod_{j\in p(s,t)}\frac{1}{k_j-1}, 
\end{equation}

Gallotti et al.~\cite{gallotti2016limits} quantified the amount of information an individual needs to travel along the shortest path between any given pair of metro stations, in the single-layer metro networks for 15 large cities. They found that this information has an upper bound of the order of 8 bits, corresponding to approximately 250 connections between different routes. Further, studying several among the largest multilayer transport networks (metro/buses/light rail), Gallotti et al.~\cite{gallotti2016limits} showed that the amount of information necessary to know to travel between any two points exceeds the identified cognitive limit of 8 bits in 80\% the cases, suggesting multilayer networks are too complex for individuals to navigate easily.

\subsubsection{Individual multi-modal behaviour.}
In recent years, data collected via smart travel cards has dramatically improved our ability to characterise multimodal behaviour in urban transport systems, overcoming some of the limitations related to collecting and analyzing survey data~\cite{chen2016promises,zannat2019emerging}. Smart-card automated fare collection systems allow passengers to make journeys involving different transport modes using magnetic cards and automatic gate machines. As these systems identify and store the location and time when individuals board and, in some cases, exit public transport, they collect accurate descriptions of individual travel~\cite{pelletier2011smart}. Concurrently, advancements were made possible by the development of methodologies allowing to identify typical travel patterns~\cite{ma2013mining}.

\paragraph{Route choices.} 
Smart-card data has allowed to quantify how individuals navigate multilayer networks. One of the key findings is that individuals do not choose optimal paths (those with shortest travel time), especially when the system is congested. Focusing on the bus and subway trips of 2.4 million passengers in Shenzen (China), Zheng et al.~\cite{zheng2018coupling} studied the coupling (see eq. \ref{eq:coupling}) between the bus and subway layers. In contrast to previous studies~\cite{strano2015features}, Zheng et al.~\cite{zheng2018coupling} characterized the coupling $\lambda$ using passenger behaviour rather than structural properties of the multilayer network. Under their definition, the \emph{coupling} between layers is the fraction of multimodal trips actually undertaken by passengers, rather than the fraction of multimodal shortest paths (see eq.\ref{eq:coupling}). Zheng et al. found that this `behavioural' coupling correlates weakly with the empirical speed ratio measured between the two layers over time, implying that passengers choose unimodal trips even when multimodal trips may be preferred because one of the two layers is congested. This finding highlights that the speed ratio of different network layers, which was regarded as a key factor in determining coupling strength~\cite{strano2015features,chodrow2016congestion}, may have a negligible effect on travelers’ route selections, possibly because passengers do not have a full view of the status of traffic. Instead, Zheng et al.~\cite{zheng2018coupling} showed that the coupling between layers is generated by long-distance trips originating from nodes served by a single transport layer.


\paragraph{Power-law distribution of displacements} 
The availability of large-scale data sources has revealed that individual mobility patterns display universal properties. One key finding is that the distribution $P(\Delta r)$ describing the probability of travelling a given distance $\Delta r$ is characterized by a power-law tail $P(\Delta r)\sim \Delta r ^{-\beta}$, with $1 \leq \beta \leq 2$~\cite{gonzalez2008understanding,barbosa2018human}. This finding is consistent across a range of studies that used different data sources\footnote{see Ref.~\cite{alessandretti2017multi} for an extensive review}. It was recently shown that these observed scaling properties result from the aggregation of movements within and across characteristic spatial scales, corresponding to the sizes of buildings, neighbourhoods, cities and regions~\cite{alessandretti2020scales}. Further, the emergence of scaling properties was associated to the use of multimodal transportation: Zhao et al.~\cite{zhao2015explaining} used GPS data to show that mobility using a single mode can be approximated by a lognormal distribution, but the mixture of the distributions associated with each modality generates a power-law; Gallotti et al.~\cite{gallotti2016stochastic} found that a simple model where individual trajectories are subject to changes in velocity generates a distribution of displacements with a power-law tail. In fact, individuals using multimodal infrastructure are subject to drastic changes in velocity. Varga et al.~\cite{varga2016further} showed that the travel speed $v$ increases with travel distance according to the power-law functional form $v \sim r ^\alpha$, where $\alpha \approx 0.5$. This dependence is due to the hierarchical structure of transportation systems and the fact that waiting-times (parking, take-off, landing, etc) decrease as a function of trip distance. 


\section{Open data \& Tools \label{sec:datatools}}

Together with other fields, urban and transportation science are becoming more open, increasingly relying on datasets and computational tools freely available to the scientific community. In this section we highlight some of these openly available datasets and computational resource developed for the analysis of multimodal transport networks.

\subsection{Data}
During the last years new datasets have been made publicly available either from the public sector or from crowd-sourced data, allowing to go beyond simulations and synthetic data to understand how urban dynamics and mobility unfold, providing a better picture of the world.

In the last years, the study of transportation networks has benefited from the development of OpenStreetMap~\cite{OpenStreetMap}, an open-source collaborative project focused on collecting and sharing world-wide high-quality spatial data~\cite{haklay2010openstreetmap,girres2010quality,ferster2019openstreetmap,barbosa2018human}. Data extracted from OpenStreetMap allows to build and analyse several transportation networks, of which the most common are street networks~\cite{boeing2020multiscale,boeing2020world}. Data from OpenStreetMap is further useful to obtain information about infrastructures, such as subways and railways. All such data can be combined to build multimodal transport networks. Gil.~\cite{gil2015multimodal} obtained multimodal data from OpenStreetMap and built a multiplex network for the Randstad region of the Netherlands linking the layers through the intersections between transport modes. More recently, Natera~\cite{natera2019data} followed the same approach to analyse the multiplex transport of fifteen cities in different development stages, including London, Los Angeles, and Mexico City.

As the transportation network of a city also encodes a temporal dimension, it is important to take into account the frequency of buses, tramways, and subways when modeling transportation and mobility. To capture these dynamics, Gallotti et al.~\cite{gallotti2015temporal} constructed and shared the temporal network of public transport in Great Britain. This is a large dataset, where links, associated to flows from one location to another, only exist at specific times. Link information has to be properly combined to compute travel time from origin and destination, highlighting the importance of synchronization of the different transportation models. An interesting feature of this dataset is that it not only contains the public transport layers of several cities, but also connections among them, for instance through coaches, planes, and ferries operating at the national level.

In general, the use of timetables and transit feeds has been enabling researchers to capture with increasing accuracy the dynamics of public transportation systems. A large collection of GTFS feeds in multiple locations has been collected and made freely available as a webpage~\cite{transitfeeds}. These datasets include stops, routes and timetables of public transport in multiple cities and providers from $667$ locations around the world. As shown by Aleta et al.~\cite{Aleta2017Multilayer}, these data can be used to analyze the public transportation as a multiplex network, considering each bus line and/or transport provider as one layer. Using the data from GTFS feeds, Kujala et al.~\cite{kujala2018collection} built and published a collection of 25 urban public transport networks covering cities from North America, Europe, and Oceania. This dataset is peculiar as it also includes the pedestrian layer of the cities, and public transport is further differentiated across the different public transport modes. 

More recently, Tenkanen et al.~\cite{tenkanen2020travel} published an accurate multimodal dataset for the Helsinki region in Finland. This dataset includes multiple transport modes, such as walking, cycling and public transportation options. To calculate the travel times Tenkanen et al.~\cite{tenkanen2020travel} use a door-to-door principle. This means that travel time and distance are calculated considering every step of a journey, including walking legs and transfers between vehicles. An important feature of this dataset is the inclusion of travel time matrices for three distinct years, 2013, 2015 and 2018. This is a rare occasion to compare how travel times changed over the years, allowing a characterization of the evolution of human mobility.

Concerning mobility, the Geolife dataset~\cite{zheng2011geolife} consists of GPS trajectories collected by Microsoft Research Asia for 178 users in a period of over four years (from April 2007 to October 2011). 69 users labeled their trajectories with the corresponding transportation mode, such as driving, taking a bus, riding a bicycle and walking. As such, the GeoLife data has allowed to investigate mobility behaviours using different transport modes~\cite{zhao2015explaining}.

The data described above are freely available, and represent an opportunity for further data-driven investigations of multimodal transportation networks.

\subsection{Tools}

Over the years computational tools have become more and more important for studying urban systems, and in particular transportation networks\footnote{For an overview of the available tools in geographic analysis in transport planning see~\cite{lovelace2021open}}. 

Multiple tools allow to work with graphs. A few examples of freely available softwares and tools are: \textit{Networkx}~\cite{hagberg2008networkx}, \textit{igraph}~\cite{csardi2006igraph}, and \textit{graph-tool}~\cite{peixoto2014graph-tool}. These tools are freely available and constantly updated over time relying on contributions from an engaged community. Although these tools serve a general purpose, they can also be used for the study of transportation networks.

Multiple tools were developed to obtain data on transportation and multimodal infrastructures. One of the best known is \textit{OSMnx}~\cite{boeing2017osmnx}, a Python package that downloads street networks from OpenStreetMap into Python objects. \textit{OSMnx} can further be used to download other transportation networks, and build its multimodal transport networks. 

Another reliable Python library to read data from OpenStreetMap and extract transportation networks is \textit{Pyrosm}~\cite{tenkanen2020pyrosm}. Differently from \textit{OSMnx}, \textit{Pyrosm} reads the data directly from OpenStreetMap's Protocol Buffer Format files (*.osm.pbf), while OSMnx downloads the data from the Overpass API~\cite{overpass}. For this reason \textit{Pyrosm} is a particularly good alternative when working with large urban areas, states, and even countries, while OSMnx typically offers a more precise way to collect data from specific points in a city.

To work with public transportation data Google developed \textit{transitfeed}~\cite{google2020gtfs}, a Python library to parse, validate and build GTFS files. This tool is particularly useful to those interested in the manipulation of the raw data. However, to convert the data into a network, some additional steps are needed. An alternative to read the GTFS feeds and directly extract its transportation network is \textit{Peartree}~\cite{butts2021peartree}, a Python library allowing to convert GTFS feed schedules into the corresponding directed network graph.

\textit{Movingpandas}, developed by Graser~\cite{graser2019movingpandas} is a Python package that provides trajectory data structures and functions for the analysis and visualisation of mobility data. In a similar sense, and also developed in Python, \textit{scikit-mobility}~\cite{pappalardo2021scikitmobility} is a library that implements a framework for analyzing statistical patterns and modeling mobility, including functions for estimating movement between zones using spatial interaction models, and tools to asses privacy risks related to the analysis of mobility datasets.

The aforementioned tools were not built specifically with the purpose to work with multiplex networks. To cover this need, specific libraries have been developed. A first example is \textit{muxViz} by de Domenico et al.~\cite{dedomenico2015muxviz}, a stand-alone front-end tool which allows the computation of several multilayer measures, from centrality to community detection. \textit{MuxViz} is also an advanced visualization tool, providing an effective way to display edge-colored multigraph or multislice networks. 

Several software options are available in Python, often built on top of NetworkX. A library originally designed for the study of multilayer networks, which can be easily adapted to multimodal networks, is \textit{MAMMULT} by Nicosia et al.~\cite{nicosia2015mammult}. This library contains a collection of algorithms to analyze and model multilayer networks. The functions included in the collection cover a wide range of applications from structural properties, such as node, edge, and layer basic properties, to the analysis of dynamics on multilayer networks, such as random walks. 

Another example is \textit{multiNetX}~\cite{kouvaris2015pattern}. This library extends Networkx allowing the creation of undirected weighted and unweighted multilayer networks from Networkx objects. Once the multilayer networks are built, the library focuses on the spectral properties of the corresponding adjacency or Laplacian matrices. This tool also provides nice visualization tools improving from Networkx, allowing the user to better visualize multilayer dynamics through coloring the nodes and links over time. 

A more recently developed Python library, not relying on Networkx, is \textit{Pymnet}~\cite{kivela2018pymnet}. The package handles general multilayer networks, including multiplex networks with temporal variables. For this reason, it is possible to use it for the analysis of multimodal urban transport networks that incorporate transit schedules. This library also includes multiple network analysis methods, transformations, and models to analyze and visualize multilayer networks. Another alternative is the \textit{multinet} library~\cite{magnani2020multiplex}, available both in Python and R. This package provides tools to work with multilayer networks, including community detection and visualizations. When visually working with multilayer networks it is important to account for principles of visualization and cognitive overload~\cite{rossi2015towards}. Another option in R is \textit{multiplex} developed by Rivero et al.~\cite{rivero2020algebraic}. This library offers multiple functions to work with matricial representations and visualization of multilayer networks. 

Finally, tools such as \textit{mapbox}~\cite{mapbox}, \textit{carto}~\cite{carto}, \textit{kepler.gl}~\cite{kepler2021kepler}, and \textit{studio unfolded}~\cite{unfolded}, built on top of OpenStreetMap~\cite{OpenStreetMap}, allow to create and share geospatial interactive web visualizations. While these tools have not been designed to work specifically with networks, it is possible to leverage their geospatial visualization capabilities to create appealing visualization of urban systems.

All the aforementioned tools are available freely online, with an open source code open to edits, collaborations and improvements. Most tools for multilayer network analysis currently serve a general purpose and have not been designed to support features for multimodal transportation networks in particular. Given the growing interest in this topic, we anticipate future open source libraries built specifically for multimodal urban data.

\section{Conclusion \label{sec:conclusions}}

In this review we discussed the state-of-the art in the field of multimodal mobility and multilayer transport networks from a complexity science perspective, focusing on urban environments. On one hand we covered the science of the \emph{dynamics} of mobility: How do people move? Which forms of transportation do they use? How do they find their paths or switch between modes? On the other hand these dynamics take place on an underlying (infra)\emph{structure} which can be well modeled by multilayer networks. In this context, a number of mathematical metrics have been developed in network science recently which allow the rigorous study of the topic. Parallel to the methodological developments we have witnessed a spur of new computational tools -- many of them open source -- and datasets which considerably facilitate and boost further research on the topic. Despite an explosion in geospatial data collection, it is still relatively difficult to access spatio-temporally fine-grained -- and appropriately anonymized~\cite{de2013unique} -- mobility data openly. Such high-quality data are in danger of being siloed in by commercial stakeholders, obstructing transparent research on the topic. We must therefore push for the implementation of better systems by governments, academia, and industry to recognize and promote efforts for making data sets and tools openly available by and for researchers~\cite{stodden2016enhancing,lovelace2021open}. 

The increasing availability of mobility data, concurrently with the continuous growth and developments of urban transport infrastructures are raising new research challenges. A first critical issue relates to the modeling of shared mobility services, such as shared bicycles and vehicles~\cite{shaheen2016mobility}. Multimodal frameworks that integrate shared services with traditional public and private transport infrastructures are becoming necessary to ensure real-time and user-centered solutions for planning, forecasting and managing services, while increasing safety, reducing congestion and emissions. A second important area focuses on investigating at scale the decision-making processes underlying individual transport mode choice and routing behaviour leveraging the increasingly available high-resolution individual traces. We anticipate that this new understanding will be key to describe how microscopic decision-making processes contribute to the emergence of collective mobility flows in multimodal systems.  

Despite the currently exploding research on multimodal mobility, there exists a wide frontier of topics to tackle and new approaches to explore. Important advances on multimodal mobility and transportation have been shown to be interdisciplinary, and have clearly benefitted from the large variety of scientific fields and practices. Indeed, synergies between disciplines such as urban planning, geoinformatics, computer science, or physics have increased in the last few years, giving rise to new interdisciplinary approaches such as a Science of Cities or Urban Data Science. We envision that research on multimodal mobility and transportation will maintain a highly interdisciplinary character in the future. For example, the study of human mobility has recently benefited from novel scientific advances from other fields such as deep learning. Luca et al.~\cite{luca2020deep} offers a comprehensive overview of the topic and its applications to human mobility, surveying data sources, public datasets, and deep learning models, and we anticipate the possibility that this area will soon make an impact in unveiling new features of multimodal mobility and transportation. 

Understanding mobility and its underlying infrastructure is of paramount importance for developing sustainable urban transport, as urban mobility relies on the central role of public transport modes and the interconnection between public transportation with other mobility infrastructures. In the next chapter we propose the Overlap Census to capture such interconnections between mobility options in a city. The study and understanding of multimodal mobility and the relations between different transport modes disregards the quality of the different layers. To be able to account for such layer quality we will focus our attention to the most underdevelopment ones, the bicycle layer and the pedestrian one. In Chapter \ref{ch:BikeGrowth} we show the underdeveloped and fragmented state of the bicycle infrastructure layer in multiple cities, while proposing an algorithmic approach to improve its connectivity. Finally, in Chapter \ref{ch:LQI} we focus on the pedestrian layer, and use routing methods on the single layer to compute accessibility metrics to amenities as a proxy to capture the quality of life in urban environments.