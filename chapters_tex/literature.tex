\chapter{Related Work}\label{ch:litReview}

The primary aim of this section is to offer a comprehensive review of multimodal transportation and mobility research focusing on recent complex systems approaches. In such approaches, the city is studied as a complex system \cite{batty2013new,lobo2020urban}, in which especially urban transport infrastructure, such as streets, sidewalks, bicycle lanes and public transportation systems can be well modelled and understood using methods from network science. From this perspective, single-layer spatial networks, especially transportation networks, have been widely studied~\cite{lin2013complex,barthelemy2011spatial,ding2019application}, finding different topological properties~\cite{jiang2004topological,cardillo2006structural,barthelemy2008patterns,batty2008size,barthelemy2011spatial,strano2013comparative,louf2014typology,boeing2020multiscale}, distribution of centrality metrics~\cite{crucitti2008centrality,boeing2018planarity,kirkley2018structural}, and network growth processes~\cite{makse1995growth,strano2012evolution}. Further topics studied include impacts of the street networks on pedestrian volume \cite{hajrasouliha2015connectivity}, accessibility and vitality of cities~\cite{denadai2016death,biazzo2019accesibility,natera2019walkability}, and resilience and growth of different transportation networks~\cite{baggag2018resilience,ferretti2019resilience,natera2020growth}. %Despite the many successes, network science should be applied to transportation systems with care \cite{zanin2018studying}. 

The most recent of these approaches can be seen as the beginning of the emerging field of Urban Data Science, which exploits large-scale new urban data sets with tools combining geoinformatics, data and network science \cite{organizers2019roundtable,resch2019hds}.

In this section we focus on the combined use of single-layer networks as multilayer networks to characterize the multimodal transportation infrastructure of cities and the human mobility taking place on them. We follow the primal approach to networks \cite{porta2006primal}, where streets and mobility infrastructure constitute the network links, and intersections (bus stops, subway stations, etc.) constitute the nodes of the network.

The remainder of this section is arranged as follows. Section~\ref{sec:multilayernetworks} introduces the mathematical concept of multilayer networks and related theoretical research underlying network science approaches to the topic. %In Section~\ref{sec:multimodalinfrastructures} we cover urban infrastructures, measures to quantify their multimodality and models to reproduce the interconnected structures of real-world transportation systems. In Section~\ref{sec:multimodalmobility} we focus on mobility, flows and navigation across these multimodal systems, and implications for transportation choices. In Section ~\ref{sec:datatools} we cover the relevant open datasets and the main software tools which can be used to analyse multimodal transportation systems. We conclude with an outlook and a summary of open questions for the research community in Section~\ref{sec:conclusions}. 

\section{Multilayer networks: a framework for multimodality}\label{sec:multilayernetworks}

Over the last decades, networks have emerged as a versatile tool to understand, map and visualise the interconnected architecture of a wide range of complex systems~\cite{albert2002statistical,dorogovtsev2002evolution, newman2003structure, boccaletti2006complex}, in particular spatially-embedded ones~\cite{barthelemy2011spatial}. Formally, a network -- or graph -- $\mathcal G = (\mathcal N, \mathcal L)$ consists of a set of nodes $\mathcal N$, and a second set $\mathcal L$ of edges, describing connections among unordered pairs of elements of the first set. This information can be conveniently stored into an adjacency matrix $A = \{a_{ij}\}$, where $i=1, \dots, N$ are the nodes, and $a_{ij}=1$ if there is a link between nodes $i$ and $j$, $a_{ij}=0$. In transportation systems~\cite{lin2013complex}, nodes can represent the stations of a network, and links direct connections between them. The adjacency matrix can also include weights $W = \{w_{ij}\}$, where $w_{ij}$ are positive real numbers, for instance describing how strongly connected two nodes are. For spatial systems, weights are often taken as the reciprocal of the distance between two nodes, or the time it takes to travel from one to another, i.e. $w_{ij}=1/d_{ij}$ or $w_{ij}=1/t_{ij}$.

More recently, network scientists have put a lot of effort in characterising the structure of systems which are formed by different interconnected networks. Also widespread in social and biological networks, these structures are natural for transportation systems. Think for instance of the largest transportation hubs in worldwide cities, where stations are routinely served by bus, underground and railway infrastructures.

Indeed, most urban transportation systems systemically rely on the interplay between different mean of transportation. These systems can be conveniently described by \textit{multiplex} or \textit{multilayer} networks. Here we introduce the so-called \textit{vectorial} formalism for multilayer networks~\cite{boccaletti2014structure, battiston2014structural}, widely used in most papers on multimodal transportation. We note that an alternative description can be provided by a more mathematically involved \textit{tensorial} framework~\cite{dedomenico2013mathematical, kivela2014multilayer}.  

In multilayer networks, links of different types, describing for instance a different mean of transportation, are embedded into different \textit{layers}. Each layer $\alpha$, $\alpha = 1, \ldots, M$, is described by an adjacency matrix 
$ W^{[\alpha]} = \{w_{ij}^{[\alpha]}\}$. In a multimodal urban transportation networks with three layers, $\alpha=1$ can represent the bus network, $\alpha=2$ the underground network, and $\alpha=3$ the urban railway network. The full transportation system $\mathcal M$ can be described as $\mathcal W = \{W^{[1]}, \ldots,  W^{[M]}\}$. Nodes $i=1, \dots, N$ are labeled in the same order in all networks. 

In the case of transportation networks, identifying nodes of different networks (urban location) as the same station might not provide the most complete description of the multimodal network. Think for instance of the largest stations in mega-cities, like King's Cross - St. Pancras in London, Grand Central Station in New York or Hongqiao transportation hub in Shanghai. All of those are identified by a unique location (node index) $i$ across the different transportation layers. However, sometimes switching from one mean of transportation to another within the same station might require a non negligible fraction of time and effort, given the complexity and size of the overall infrastructure. 

For this reason, it is often relevant to complement the description of the \textit{intra-layer} connections present in the system, with \textit{inter-layer} links associated to the cost, physical distance or time required to switch layers. Inter-layer links between layers $\alpha$ and $\beta$ at a node $i$ can be encoded through the inter-layer matrix $C_i=\{c_i^{[\alpha \beta]} \}$, and all such inter-layer connections can be stored in the vector $ C = {C_1, \ldots, C_N}$. In this case, the full multiplex structure of the system is described by taking into account both intra-layer and inter-layer connectivity, hence $\mathcal M = ( W,  C)$. Inter-links may be neglected for many measures focusing on diversity~\cite{battiston2014structural}, as well as correlations~\cite{nicosia2015measuring} across the layers of the systems, relevant to assess the different roles and geographical spanning of the different mean of transportations of a multilayer network. 

% Multilayer networks are a natural framework for multimodal networks. Indeed, one of the pioneering works introducing the framework and concept of `layered complex networks'~\cite{kurant2006layered} explicitly focused on the case of transportation systems, where a first layer encoded the physical infrastructure of the system, and the second one described the flows on such infrastructure. Other early works on the topic also dealt with interconnected systems at the wordwide level, focusing on different modes of transport such as the multiplex airline networks~\cite{cardillo2013emergence}.

% Noticeably, multimodal infrastructures seem to possess exclusive characteristics, different from other multilayer networks. For instance, when tools to assess the redundancy of the different layers are considered, transportation networks are often found to be irreducible~\cite{dedomenico2015structural}. Differently from many biological systems, where layers often duplicate information to guarantee the interconnected system a high level of robustness, the layers of a multiplex transportation systems are purposedly engineered to be different, in order to maximise efficiency~\cite{latora2001efficient}. As a byproduct of this feature, multimodal systems are also often highly fragile~\cite{buldyrev2010catastrophic}, and sensitive to disruptions or failures of a single infrastructure~\cite{dedomenico2014navigability}. For the reader interested in further material on the topic, we refer to the early reviews~\cite{boccaletti2014structure, kivela2014multilayer} and textbook~\cite{bianconi2018multilayer} covering the field. Ref.~\cite{aleta2019multilayer} provides a more recent eye-bird view of the field. A thorough review of the measures and models used to analyse such systems can be found in Ref.~\cite{battiston2017new}, whereas Ref.~\cite{dedomenico2016physics} gives a theoretical overview of spreading and diffusive processes on such systems. In the following sections of this review, we focus on findings of more direct relevance to the research community working with multimodal transportation and urban mobility. 

% To include: section Multimodal