\chapter{Related Work: Multimodal transportation and mobility in urban networks}\label{ch:litReview}

In this chapter we provide a comprehensive overview of complex systems approaches to multimodal transportation and mobility in urban networks. We cover multiple approaches that had been used to model cities using the multiplex framework, measures to quantify centralities and changes in the given networks and layers. After reviewing the mobility infrastructure and measures, we move our focus to the mobility dynamics on top of those layers, covering the study of urban mobility dynamics in public transport systems and at individual level. Finally we conclude the chapter with an overview of the available data and computational tools for the study of urban mobility networks\footnote{A stand alone of this chapter is currently being prepared as a paper, to be submitted to Transportation Reviews. No preprint is yet available}.

\section{Cities as complex systems}
In this first chapter we offer a comprehensive review of multimodal transportation and mobility research focusing on recent complex systems approaches. In such approaches, the city is studied as a complex system \cite{batty2013new,lobo2020urban}, in which especially urban transport infrastructure, such as streets, sidewalks, bicycle lanes and public transportation systems can be well modelled and understood using methods from network science. From this perspective, single-layer spatial networks, especially transportation networks, have been widely studied~\cite{lin2013complex,barthelemy2011spatial,ding2019application}, finding different topological properties~\cite{jiang2004topological,cardillo2006structural,barthelemy2008patterns,batty2008size,barthelemy2011spatial,strano2013comparative,louf2014typology,boeing2020multiscale}, distribution of centrality metrics~\cite{crucitti2008centrality,Boeing2020Planarity,kirkley2018structural}, and network growth processes~\cite{makse1995growth,strano2012evolution}. Further topics studied include impacts of the street networks on pedestrian volume \cite{hajrasouliha2015connectivity}, accessibility and vitality of cities~\cite{denadai2016death,biazzo2019accesibility,natera2020walkability}, and resilience and growth of different transportation networks~\cite{baggag2018resilience,ferretti2019resilience,natera2020growth}. %Despite the many successes, network science should be applied to transportation systems with care \cite{zanin2018studying}. 

The most recent of these approaches can be seen as the beginning of the emerging field of Urban Data Science, which exploits large-scale new urban data sets with tools combining geoinformatics, data and network science \cite{organizers2019roundtable,resch2019hds}.

In this section we focus on the combined use of single-layer networks as multilayer networks to characterize the multimodal transportation infrastructure of cities and the human mobility taking place on them. We follow the primal approach to networks \cite{porta2006primal}, where streets and mobility infrastructure constitute the network links, and intersections (bus stops, subway stations, etc.) constitute the nodes of the network.

The remainder of this section is arranged as follows. Section~\ref{sec:multilayernetworks} introduces the mathematical concept of multilayer networks and related theoretical research underlying network science approaches to the topic. In Section~\ref{sec:multimodalinfrastructures} we cover urban infrastructures, measures to quantify their multimodality and models to reproduce the interconnected structures of real-world transportation systems. In Section~\ref{sec:multimodalmobility} we focus on mobility, flows and navigation across these multimodal systems, and implications for transportation choices. In Section ~\ref{sec:datatools} we cover the relevant open datasets and the main software tools which can be used to analyse multimodal transportation systems. %We conclude with an outlook and a summary of open questions for the research community in Section~\ref{sec:conclusions}. 

\section{Multilayer networks: A framework for multimodality}\label{sec:multilayernetworks}

Over the last decades, networks have emerged as a versatile tool to understand, map and visualize the interconnected architecture of a wide range of complex systems~\cite{albert2002statistical,dorogovtsev2002evolution, newman2003structure, boccaletti2006complex}, in particular spatially-embedded ones~\cite{barthelemy2011spatial}. Formally, a network -- or graph -- $\mathcal G = (\mathcal N, \mathcal L)$ consists of a set of nodes $\mathcal N$, and a second set $\mathcal L$ of edges, describing connections among unordered pairs of elements of the first set. This information can be conveniently stored into an adjacency matrix ${A=a_{ij}}$, where $i=1, \dots, N$ are the nodes, and $a_{ij}=1$ if there is a link between nodes $i$ and $j$, $a_{ij}=0$ if there is no link between $i$ and $j$. In transportation systems~\cite{lin2013complex}, nodes can represent the stations of a network, and links direct connections between them. The adjacency matrix can also include weights $W=w_{ij}$, where $w_{ij}$ are positive real numbers, for instance describing how strongly connected two nodes are. For spatial systems, weights are often taken as the reciprocal of the distance between two nodes, or the time it takes to travel from one to another, i.e. $w_{ij}=1/d_{ij}$ or $w_{ij}=1/t_{ij}$.

More recently, network scientists have put a lot of effort in characterizing the structure of systems which are formed by different interconnected networks. Also widespread in social and biological networks, these structures are natural for transportation systems. Think for instance of the largest transportation hubs in worldwide cities, where stations are routinely served by bus, underground and railway infrastructures.

Indeed, most urban transportation systems systemically rely on the interplay between different mean of transportation. These systems can be conveniently described by \textit{multiplex} or \textit{multilayer} networks. Here we introduce the so-called \textit{vectorial} formalism for multilayer networks~\cite{boccaletti2014structure, battiston2014structural}, widely used in most papers on multimodal transportation. We note that an alternative description can be provided by a more mathematically involved \textit{tensorial} framework~\cite{dedomenico2013mathematical, kivela2014multilayer}.  

In multilayer networks, links of different types, describing for instance a different mean of transportation, are embedded into different \textit{layers}. Each layer $\alpha$, $\alpha = 1, \ldots, M$, is described by an adjacency matrix 
$ W^{[\alpha]} = w_{ij}^{[\alpha]}$. In a multimodal urban transportation networks with three layers, $\alpha=1$ can represent the bus network, $\alpha=2$ the underground network, and $\alpha=3$ the urban railway network. The full transportation system $\mathcal M$ can be described as $\mathcal W = \{W^{[1]}, \ldots,  W^{[M]}\}$. Nodes $i=1, \dots, N$ are labeled in the same order in all networks. 

In the case of transportation networks, identifying nodes of different networks (urban location) as the same station might not provide the most complete description of the multimodal network. Think for instance of the largest stations in mega-cities, like King's Cross - St. Pancras in London, Grand Central Station in New York or Hongqiao transportation hub in Shanghai. All of those are identified by a unique location (node index) $i$ across the different transportation layers. However, sometimes switching from one mean of transportation to another within the same station might require a non negligible fraction of time and effort, given the complexity and size of the overall infrastructure. 

For this reason, it is often relevant to complement the description of the \textit{intra-layer} connections present in the system, with \textit{inter-layer} links associated to the cost, physical distance or time required to switch layers. Inter-layer links between layers $\alpha$ and $\beta$ at a node $i$ can be encoded through the inter-layer matrix $C_i=c_i^{[\alpha \beta]}$, and all such inter-layer connections can be stored in the vector $ C = \{C_1, \ldots, C_N\}$. In this case, the full multiplex structure of the system is described by taking into account both intra-layer and inter-layer connectivity, hence $\mathcal M = (W,  C)$. Inter-links may be neglected for many measures focusing on diversity~\cite{battiston2014structural}, as well as correlations~\cite{nicosia2015measuring} across the layers of the systems, relevant to assess the different roles and geographical spanning of the different mean of transportation of a multilayer network. 

Multilayer networks are a natural framework for multimodal networks. Indeed, one of the pioneering works introducing the framework and concept of `layered complex networks'~\cite{kurant2006layered} explicitly focused on the case of transportation systems, where a first layer encoded the physical infrastructure of the system, and the second one described the flows on such infrastructure. Other early works on the topic also dealt with interconnected systems at the worldwide level, focusing on different modes of transport such as the multiplex airline networks~\cite{cardillo2013emergence}.

Noticeably, multimodal infrastructures seem to possess exclusive characteristics, different from other multilayer networks. For instance, when tools to assess the redundancy of the different layers are considered, transportation networks are often found to be irreducible~\cite{dedomenico2015structural}. Differently from many biological systems, where layers often duplicate information to guarantee the interconnected system a high level of robustness, the layers of a multiplex transportation systems are purposely engineered to be different, in order to maximize efficiency~\cite{latora2001efficient}. As a byproduct of this feature, multimodal systems are also often highly fragile~\cite{buldyrev2010catastrophic}, and sensitive to disruptions or failures of a single infrastructure~\cite{dedomenico2014interconnected}. For the reader interested in further material on the topic, we refer to the early reviews~\cite{boccaletti2014structure, kivela2014multilayer} and textbook~\cite{bianconi2018multilayer} covering the field. Ref.~\cite{aleta2019multilayer} provides a more recent eye-bird view of the field. A thorough review of the measures and models used to analyse such systems can be found in Ref.~\cite{battiston2017new}, whereas Ref.~\cite{dedomenico2016physics} gives a theoretical overview of spreading and diffusive processes on such systems. In the following sections of this review, we focus on findings of more direct relevance to the research community working with multimodal transportation and urban mobility. 


\section{Multimodal infrastructures}\label{sec:multimodalinfrastructures}

Multiple approaches have been followed to study the structure of cities, and since the 1950s fields such as Architecture, Urbanism and Transport Planning have grown a large body of literature studying the form and structure of cities and their transport systems. As cities grow and add different transportation modes, understanding the transportation infrastructure and its interconnected nature is crucial to fully capture real-world patterns of urban mobility. With the growth of the fields of Complex Systems and Network Science new tools and models have been develop to study the complexity behind urban systems, and specifically mobility infrastructure.

In particular, multimodal urban infrastructures can be represented as multilayer networks, in which each layer $\alpha$ represents a mobility infrastructure (e.g. subway, light railway, bus service, pedestrian or bicycle infrastructure), the set of nodes $\mathcal{N}$ are locations (e.g. bus stops, intersections, subway stations), and the set of edges $\mathcal{L}$ in layer $\alpha$ are the infrastructure links between nodes in the same layer (e.g. subway lines, bus routes, bike lanes, see also Section II). Modeling infrastructures is of great importance for the understanding of how urban systems work, and the design of new sustainable mobility options.

How do transportation layers are coupled and grow?
In the following we review the main models used to characterize multimodal urban transportation systems from a complex systems perspective, from understanding multimodality in the city to simulate and test different scenarios with realistic data.
% In the following we will review the scientific literature on the modelling (Section~\ref{sec:modelinginsrastructure}) and the characterization of multimodal urban transportation systems from a complex systems perspective (Section~\ref{sec:measuresinfrastructure}).


\subsection{Modeling urban infrastructure}\label{sec:modelinginsrastructure}

Transportation agencies increasingly rely on modeling approaches from the network science community to describe transportation options in a city and evaluate the development of urban infrastructure.

One of the first contributions to the modeling of multimodal urban infrastructures was provided by de Cea et al.~\cite{decea2005equilibrium}. There, the authors point out that most of the models from the transport community used to plan and simulate the effects of new transportation options failed to consider congestion associated with transit modes. This shortcoming is particularly relevant when the models are used for infrastructure development and predict transportation equilibria in future years. 

The model proposed by de Cea et al.~\cite{decea2005equilibrium} takes the road network as the base layer. There, links have an average operating cost that takes into account different mobility options (e.g. cars, taxi, etc.). For every public transportation mode a new layer is defined, with their unique nodes and links. For these layers, the cost function of the public transportation links depends on the combined effect of travel, waiting, and transfer time.

The model considers the existence of combined trips (e.g. car/metro, bus/metro, etc) and looks for an equilibrium condition under the assumption that every user chooses her route to minimize their average operation cost (Wardrop’s first principle). This means that at equilibrium, only non-congested routes have a minimum cost, while those without flow have represent a more costly option. Travel time might be affected by the interplay of different transportation means. For instance, vehicle flow over the road network may induce longer travel times in the bus network. Similarly, a passenger might decide not to take the subway if it is too crowded. 
 
The authors show that their model is able to find the equilibrium in the trips between origin and destination in a toy model, and can be successfully applied to real-world scenarios. For instance, a rich version of this model (which comprised of 13 user classes, 11 transport modes, and 450 zones) successfully informed the planning of the new metro line 5 in Santiago (Chile).

A similar philosophy was deployed by Li et al.~\cite{li2007parkride} which considered a transportation infrastructure of cars, combined walk-metro paths, and park-and-ride. Differently from the previous work in which the model takes in consideration the availability of routes, here the authors explicitly consider the parking availability and time spent by car commuters while looking for parking, focusing on the interplay and impact of park-and-ride (P\&R) schemes to encourage users to switch from car travel to subway and public transportation options when traveling to the cities' central area.

The model proposed by Li et al.~\cite{li2007parkride} considers the effects of traffic conditions on travel demand, and incorporates elastic demand into the model to capture commuters’ responses to traffic congestion and availability of parking supply. The responses of a user include the decision to switch to another transportation option, or to not make the trip at all. For the public transportation layer, the model also takes in consideration discomfort that may result from crowded subways. 

Through numerical simulations, the authors found that it is possible to reach an equilibrium control which prevents the emergence of traffic jams in the city center by implementing a suitable P\&R scheme, which looks at the combined effect of cost at the P\&R sites, parking availability in the city center, as well as metro fares and frequency.

While in these works the interplay  of different transportation modes was introduced, they were not yet modeled as a multilayer network. This new modeling framework was first explicitly considered for real world urban transportation infrastructure by Gil~\cite{gil2014configuration}. There, the authors proposed to use open data from Open Street Map to model the urban mobility network of a given city. The proposed model is composed of three layers. The first one is the street network, where the nodes are intersections and links are streets. This layer, accounting for private transportation, was again the reference layer with respect to the other transportation modes in the system, meaning that all other layers have to be connected to and interact with it. The second one is the public transport layer that represents the stations as nodes, and link them whenever there is a public transport service between two stations. This layer is coupled to the street layer by the stations and their closest street intersection. The authors also include in the their model land use, to measure urban accessibility.

The authors applied their framework to analyze the Randstad city-region in the Netherlands. They tested the model under different parameters and layer combinations, measuring reachability of the different city areas through closeness and betweenness centralities. By comparing with ground truth data, the authors found that the betweenness centrality in the public transportation layer is a good indicator of passenger flows.

In Aleta et al.~\cite{Aleta2017Multilayer} the authors further benefit from the richness of multilayer networks for transportation systems, highlighting two possible frameworks.
For instance, each bus or metro line could also be considered as an independent layer. This approach is useful to have a realistic model of human mobility, which takes into consideration transfer times and synchronization between single trips. Yet, it does not allow to evaluate the importance of an entire transportation mode. To do so, following the previous literature, they also aggregate all lines of the same mode and combine them in a so-called \textit{superlayer}, which are fundamental to study the interdependency and resilience of the whole system.

Using the two proposed approaches, the authors investigated the public transport systems of nine European cities. Following the first one, the authors focus on some structural features of the emerging system, such as the overlapping degree (sum of the node's degree in all layers \cite{battiston2014structural}). They found that public transport infrastructures have some universal properties, and that the maximum overlapping degree is quite similar in all the systems, even if the number of layers is different. This depends on the fact that networks are embedded in a physical space, hence imposing some bounds on the maximum number of links of each node structural constraints.

Following the second approach, Aleta et al.~\cite{Aleta2017Multilayer} investigated in details the superlayers. They found that -- surprisingly -- the nodes with the highest overlapping degree are not necessarily the ones with the highest superlayer activity (fraction of nodes that are active in $1, 2, \dots, M$ layers \cite{nicosia2015measuring}). Indeed, some transportation modes (and in particular the bus layer) have a tendency to have transportation hubs which might be disconnected to the other transportation modes, leading to high overlapping degree but low multilayer activity. 

This is relevant when considering the robustness of the whole system. As the authors pointed out, it is often easier to move a bus stop to a street nearby, even if it a local hub where multiple lines stop, than solving a disruption in a subway station. The authors also assess the importance of the superlayers based on the number of shortest paths that make use of the superlayer, a measure that we characterize later in Section~\ref{interdependence}.

The models discussed in this section shown how the multilayer networks framework has been used to investigate the structure, function and vulnerabilities of a complex transportation system. In the next section we cover some measures to quantify the multiplexity of these structures and their interconnections.

\subsection{Characterizing multimodal infrastructure}\label{sec:measuresinfrastructure}

In the previous section we saw how to model the different mobility options as multilayer networks. The empirical study of multimodal transportation infrastructure has revealed different structural properties for this specific kind of networks. But how to quantify the effectiveness of their interconnectedness? Here we discuss a number of measures used to capture the multiplexity of multimodal infrastructures, such as the importance of different nodes, the system's resilience, and the similarity between layers.

\paragraph*{Paths} 
At a more global scale, multimodality is often associated to the ability of an agent to navigate the system by using the available transportation modes. For this reason, the navigation of an agent in a transportation network can be measured by looking at the available types of paths.

A first possibility is to consider the quickest path, which neglects waiting times between transportation modes and is computed using the largest speed in the edges, and assuming a perfect synchronization between the different transportation systems. This would be the equivalent to find the shortest path between $i$ and $j$ in a weighted-single layer network. However, the transportation networks also have a temporal dimension. In order to find a path that allows for a change between two -or more- different transportation options we must find a time-respecting path, this is the shortest path between nodes $i$ and $j$ that takes into account the departures and arrivals constraints usually given by timetables and more recently by GTFs Feeds. Furthermore, for multimodal transportation networks the walking transfer time between modes has to be taken into account when computing time-respecting paths.

Yet, this approach often falls short in capturing real patterns of mobility. Indeed, transportation networks also have a temporal dimension that has to be taken into consideration. In order to find a path that allows for a change between two -or more- different transportation options we must find a time-respecting path. This is the shortest path between nodes $i$ and $j$ that considers the departures and arrivals constraints usually given by timetables and more recently by General Transit Feeds Specification (GTFS) Feeds (For an overview of possible data sources see Section~\ref{sec:datatools}). Furthermore, for multimodal transportation networks the walking transfer time between modes has to be taken into account when computing time-respecting paths.

Besides the time constraints to find a viable path between $i$ and $j$ in a multilayer transportation system, we also have to assess the validity of such a path~\cite{battista1996path, lozano2001path}, meaning that it the proposed sequence of modes is reasonable. For instance, in some cases a path composed by subway-bus-subway-private car-subway might solve the shortest path, but the presence of private transportation as an intermediate option makes it an `illogical path' and an unlikely choice for a user. The validity of transportation sequences can in general be formalized in terms of cost associated to change of transportation mode.

The task to find viable paths is one of the most important problems in urban transportation, as it has the potential to help user to find the most efficient paths in the city. Lozano et al.~\cite{lozano2001path} have proposed an efficient algorithm to find such paths when the agent establishes her limitations on the number of modal transfers.

As shown by Aleta et al.~\cite{Aleta2017Multilayer}, the contribution of the different layers of a multiplex networks to shortest paths might be very unequal. For instance, the rail systems (trams, subway) contribute to most of the shortest paths in a city, connecting distant points at a greater velocity and in straighter routes than bus or other transport modes. However, such layers have only few stations, and slower and more local transport modes often serve in a complementary role, offering a deeper coverage of the city.

\paragraph*{Spatial outreach}
The availability of different transportation modes, such as subways or tramways affects how easily it is to reach certain locations in the city. One possibility to measure this effect is to quantify the associated \textit{spatial outreach}. As defined in \cite{strano2015features}, the spatial outreach can be computed as the average Euclidean distance from node $i$ to all other nodes in the same layer $\alpha$ that are reachable within a given travel cost $\tau$. Mathematically it is defined as follows:

\begin{equation}
    L_\tau(i)=\frac{1}{N(\tau)}\sum_{j|\tau_{m}(i,j)<\tau}d_e(i,j),
    \label{eq:outreach}
\end{equation}

where $d_e(i,j)$ is the Euclidean distance between nodes $i$ and $j$, and $N(\tau)$ is the number of nodes reachable on the multilayer network within a travel cost $\tau$.

Strano et al.~\cite{strano2015features} modified the average speed (traversal time of a link) in the layers of multiplex systems to measure their effects on the corresponding travel outreach. They found that when the metro speed increases compared to the street layer speed, a clear area of high-outreach nodes emerges in the city center and around the nodes that have connections to the high-speed layer. In other words, as the velocity in layer $\beta$ increases the nodes that are closer to the interchange nodes in layer $\alpha$ improves their accessibility, implying that a person can efficiently travel from this areas to faraway places. 

This concept is similar to that of isochrones, which quantify the accessible area from a given point within a certain time threshold (e.g. What is the area that a user can reach traveling 5 minutes, in any direction, from a given point?). In the work of Biazzo et al.~\cite{biazzo2019accesibility}, the authors used this approach to measure accessibility in different urban areas computing the isochrones as a combination of public transit and pedestrian infrastructure. With this method, the authors obtained scores that capture how well a city is served by the public transit and how accessible is a specific area of the city to the rest of the city.


\paragraph*{Betweenness centrality and interdependence}
Centrality scores are one of the basic ways to characterize the relevance of the nodes in a network. In multiplex network, the overall centrality of a location or a station depends on the interplay of the different transportation options.

A simple way to compute centrality is just to evaluate a node degree, accounting for the number of locations directly connected from one part of the city. However, such a measure is not very relevant in spatial systems. For instance, in the street layer the number of possible connections from one node to the others is very constrained by the physical space, as one intersection has a limited number of streets/sidewalks that can intersect there. For this reason, different scores are typically used to assess the relevance of a city location.

One of these measures is the betweenness centrality~\cite{Freeman1977Centrality}, that in the absence of explicit mobility data, can be considered a good proxy to find areas at risk to be overcrowded and become bottlenecks in the system. 

In multiplex networks, shortest paths can go from one node to another by making use of two or more layers. We quantify this effect by measuring the  interdependence of a node $i$ as~\cite{morris2012transport,nicosia2013growing,battiston2014structural,strano2015features}

\begin{equation}
    \lambda_i=\frac{1}{N-1}\sum_{j\neq i}\frac{\psi_{ij}}{\sigma_{ij}},
    \label{eq:coupling}
\end{equation}


where $\psi_{ij}$ is the number of shortest paths between $i$ and $j$ that use edges in two or more layers, and $\sigma_{ij}$ is the total amount of shortest paths between $i$ and $j$. Node interdependence takes values in $[0, 1]$, with larger values associated to a higher coupling of the layers, while values closer to $0$ mean that most of the paths from that node to the others go through just one layer. By taking the average over all nodes $\lambda = 1/N \sum_i\lambda_i$ it is possible to obtain a single score for the whole system. 

In~\cite{Aleta2017Multilayer}, the authors have modified such measure to obtain a score for a specific layer. The layer interdependence for layer $\alpha$ is defined as:

\begin{equation}\label{eq:layer_interdependency}
    \lambda^{\alpha}=\frac{\sum_i\sum_{i\neq j}\psi_{ij}^{\alpha}}{\sum_i\sum_{i\neq j}\psi_{ij}},
\end{equation}

where $\psi_{ij}^\alpha$ describes the number of shortest paths between nodes $i$ and $j$ that use two or more layers and at least one of them is layer $\alpha$.
Note that the normalization is over the multilayer shortest paths to find the layers that are used to change to another one.

When applying this measure to multimodal transport networks, Aleta et al.~\cite{Aleta2017Multilayer} found that the metro and trams layers play an important role in concentrating shortest paths. For Madrid, the authors found that more than 40\% of the trips have at least one link in the metro layer, even if the metro layer has only 241 nodes while the bus layer has 4590 nodes. 

Similarly, the effect of the different layers can also be computed on the multilayer betweenness centrality of city locations. As cities grow and new lines and options are added into the mobility system, new interconnections between layers also appear, changing the betweenness of the different nodes. Ding et al.~\cite{ding2018traffic} studied how centralities evolved when the rail network of Kuala Lumpur grew from a tree-like structure to a more complex one. Their findings suggest that as the network grows the average shortest path in the multilayer network decreases dramatically, in this way directly affecting the betweenness centrality as new nodes are able to serve as interchange between layers, enabling new shortest paths along the system.

The results from Ding et al.~\cite{ding2018traffic} are in line with the previous findings by Strano et al.~\cite{strano2015features}, where they compared centrality distribution with and without including subway lines in London and New York. In particular, the authors show that the introduction of new layers and their interconnections plays an important factor on decentralizing congestion in the street layer, moving traffic from internal street routes and bridges to the terminal points of the subway system which might be used as interchange locations for suburban flows into the city center. More theoretical work done by Sol\'{e}-Ribalta et al.~\cite{sole-ribalta2016congestion} confirms that the main driver in traffic dynamics and congestion in multimodal transport networks is the change of paths from the least efficient layers to the most efficient one. 

\paragraph*{Resilience}
Multimodality also significantly affect the resilience of a transportation system.  Indeed, evaluating the robustness of a transport system under failures is an important task with practical implications in urban planning.

In a single layer network, the disrupture of an infrastructure (which can be described by the temporary removal of a link) often effectively makes a station or a part of the city disconnected. Imagine a transit station, in a single layer network if the links connecting the station with the rest of the system are removed, the station is inaccessible. However, if the node is part of a multimodal transportation network, this means it can still be accessed by taking advantage of the other layers. To measure the impact of multimodality on resilience de Domenico et al.~\cite{dedomenico2014interconnected} used random walks to mimic trips among locations and investigated the coverage time in the London's transportation system under different scenarios, showing that the interconnected nature of the different transport modes dramatically enhances the overall system resilience to failure compared with the single layers. 

A similar approach was followed by Baggag et al.~\cite{baggag2018resilience}, where again the coverage time of random walks was used to measure the robustness in the multimodal transportation networks of Chicago, London, New York, and Paris. To mimic realistic trips, the authors introduced several constrains on the complexity of the travel, for instance limiting the maximum number of transport modes changes. More recently Ferretti et al.~\cite{ferretti2019resilience} used the multiplex framework to model Singapore's public transportation infrastructure and test its resilience against floods in the city in different scenarios, finding that the system is extremely resilient, as it faces the first significant disruption only after the removal of $~50\%$ of it edges.

\paragraph*{Overlap census}\label{overlap}

Cities across the world have transportation networks with different degrees of multimodality and integration. How can we quantify the differences between two cities such as Mexico city and Amsterdam and reveal their differences and similarities on mobility options? The overlap census proposed by Natera et al.~\cite{natera2020multimodal} is a method that helps to answer such questions. In Chapter~\ref{ch:OverlapCensus} we cover this original contribution in detail.

The overlap census is not the first measure to compare similarities between multiplex networks and layers. Indeed, similar approaches have been developed for multiplex networks more in general. For a detail view we suggest the reader the work by \cite{nicosia2015measuring} in which the authors propose measures to capture nontrivial correlations in multiplex networks, and models to reproduce those correlations. More recently, \cite{brodka2017similarity} also presented an overview of different metrics to compute similarities between layers in multiplex networks. 


\section{Multimodal mobility}\label{sec:multimodalmobility}

Understanding urban travel is paramount for a range of real-world applications, including planning transportation~\cite{patriksson2015traffic} and designing urban spaces. Starting from the 1950s, a large body of literature in the fields of Geography and Transportation has studied how people move and use transportation technology. 

As the transport infrastructure becomes increasingly multi-modal, modelling how individuals make travel decisions in complex interconnected networks is critical. In recent years, the scientific understanding of Human Mobility has dramatically improved, also due to the widespread diffusion of mobile-phone devices and other positioning technologies, which allowed to gather large-scale geo-localized datasets of human movements and develop increasingly realistic behavioral models. Concurrently, these recent developments were made possible by the dramatic growth of the fields of Complex Systems and Network Science, which brought together ideal tools to study interconnected systems (for a comprehensive review of the recent literature stream of Human Mobility, the reader can refer to~\cite{barbosa2018human}). Despite these advancements, our understanding of multimodal mobility in urban systems remains limited, also due to the difficulties related to collecting comprehensive data across multiple transportation modalities. 

In the following, we overview the scientific literature on multi-modal mobility. While our focus will be on multi-modality, we will inevitably touch upon some of the concepts related more broadly to modelling of urban travel. In Section~\ref{mobility_1}, we briefly overview existing models, focusing on latest advances driven by the Complex Systems literature. In Section~\ref{mobility_2}, we review measures and empirical findings, with a focus on recent studies based on passively collected data sources. 

\subsection{Modeling urban mobility \label{mobility_1}}
Modelling travel demand in a multimodal system involves understanding how individuals make decisions in a constantly changing complex environment. The most diffused family of models for travel demand in the Geography and Transportation literature are the so-called \emph{four-step models} proposing that each trip results from four decisions~\cite{mcnally2000four}: whether to make a trip or not, where to go, which mode to use, and which path to take. For simplicity, these have been largely considered as independent, sequential choices. We refer the reader to the excellent review by McNally~\cite{mcnally2000four}, for a comprehensive overview about this modeling approach.

In recent years, the literature in Complex Systems has modeled travel behavior on multiplex networks using different approaches, that we briefly review in this section. It is important to remark that, also due to the lack of empirical data on the mechanisms driving human navigation, most models are based on the simplistic assumption that individuals are rational, homogeneous and have unlimited knowledge. Research based on novel data sources will be key to develop mobility models on multilayer networks that include realistic elements such as limited knowledge and cognitive limitations.

\paragraph{Random walks.}
A random walk process is a prototypical model of mobility on a network, and it is defined by a walker that, located on a given node $i$ at time $t$, hops to a random nearest neighbor node $j$ at time $t + 1$. In the case of multilayer networks, the walk between nodes and layers can be described with four transition rules accounting for all possibilities~\cite{dedomenico2014interconnected}: (i) $P_{ii}^{\alpha\alpha}$, the probability for staying in the same node $i$ and layer $\alpha$; (ii) $P_{ij}^{\alpha\alpha}$ the probability of moving from node $i$ to $j$ in the same layer $\alpha$; (iii) $P_{ii}^{\alpha\beta}$ the probability of staying in the same node $i$ while changing to layer $\beta$; (iv) $P_{ij}^{\alpha\beta}$ the probability of moving from node $i$ to $j$ and from layer $\alpha$ to $\beta$, in the same time step. These probabilities depend on the strength of the links between nodes and layers (e.g. the frequency of vehicles and the cost associated to switching layers). 

Notwithstanding their simple formulation, random walks provide fundamental hints to understand all types of diffusion processes on networks and measure the dynamical functionality of a network. 
For example, random walks processes were studied to measure the navigability of multiplex networks~\cite{dedomenico2014interconnected}. 
To this end, one can measure the coverage of the multiplex network $\rho(t)$, defined as the average fraction of distinct nodes visited by a random walker in a time shorter than $t$ (assuming that walks started from any other node in the network), and describing the efficiency of a random walk in the network exploration:

\begin{equation}\label{coverage}
    \rho(t)=1-\\\frac{1}{N^2}\sum_{i,j=1}^{N}\delta_{i,j}(0)\text{exp}[-\mathbf{P}_j(0)\mathbb{P}\mathbf{E}_i^{\dagger}],
\end{equation}

where $\mathbf{P}_j(0)$ is the supravector of probabilities at time $t=0$, the matrix $\mathbb{P}$ accounts for the probability to reach each node through any path of length $1; 2; \dots \text{or}\ t+1$, and $\mathbf{E}_i^{\dagger}$ is a supra-canonical vector i allowing to compact the notation. In the work by De Domenico et al.~\cite{dedomenico2014interconnected}, the authors provided an alternative representation of equation \ref{coverage} building upon the eigendecomposition of the supra-Laplacian. They showed that the ability to explore a multiplex is influenced by different factors, including the topological structure of each layer and the strength of interlayer connections and the exploration strategy. Further, they showed that the multilayer system is more resilient to random failures than its individual layers separately because interconnected networks introduce additional paths from apparently isolated parts of single layers, and thus enhance the resilience to random failures.

Random walks can be used to assign a measure of importance to each node in each layer, by measuring the asymptotic probability of finding a random walker at a particular node-layer as time goes to infinity, the so-called `occupation centrality'\cite{sole2016random}. In the work by Sol\'{e}-Ribalta et al.~\cite{sole2016random}, the authors provided analytical expressions for the occupation centrality in the case of multilayer networks.

\paragraph{Agent-based models accounting for congestion.} 
Several works have focused on modelling agents aiming at minimizing total travel time in a congested network~\cite{tan2014congestion,bassolas2020scaling,manfredi2018congestion,sole-ribalta2016congestion}.

In Bassolas et al.~\cite{bassolas2020scaling}, the authors developed an agent-based models describing individuals mobility through a multilayer transportation system with limited capacity. The routing protocol used by individuals for planning is adaptive with local information. In the absence of congestion, individuals follow the temporal optimal path of the static multilayer network calculated by the Dijkstra algorithm. If there are line changes, they estimate, besides the change walking penalty, an additional waiting time of half the new line period (the real waiting time will be given by the vehicles location in the line when the individual arrives at the stop). Individuals’ route is only recalculated when a congested node, whose queue is larger than the vehicle’s capacity, is reached. They investigate analytically (for simple networks) and via numeric simulations the robustness of the network to exceptional events which give rise to congestion, such as demonstration concerts or sport events. The study revealed that the delay suffered by travellers as a function of the number of individuals participating in a large-scale event obeys scaling relations. The exponents describing these relations can be directly connected the number and line types crossing close to the event location. The study suggested a viable way to identify the weakest and strongest locations in cities for organizing massive events.

In a similar way, in the work by Manfreid et al.~\cite{manfredi2018congestion} the authors introduce a limit to the nodes capacity of storing and processing the agents. This limitation triggers temporary faults in the system affecting the routing of agents that looks for uncongested paths. 


\subsection{Characterizing multi-modal mobility. \label{mobility_2}}
Traditionally, multi-modal mobility models are calibrated using data from travel surveys: \emph{Revealed Preference} surveys retrieve actual travel information from the respondents, while \emph{Stated Preference} surveys expose the travelers to various hypothetical scenarios and record their choices~\cite{arentze2013travelers}. Studies based on survey data have provided insights into how multi-modal travelers value aspects such as the different travel time components (in-vehicle time, walk time, access time, wait time...)~\cite{abrantes2011meta}, service quality~\cite{wardman2001review}, and travel costs~\cite{arentze2013travelers}, and heterogeneities across socio-demographic groups ~\cite{nobis2007multimodality}. Due to the high costs associated with data collection and inherent biases in self-reported data, these studies suffer of serious limitations, including small sample sizes, data inaccuracy and incompleteness~\cite{chen2016promises,zannat2019emerging}. Covering empirical results from travel surveys is outside the scope of this review, and we refer the reader to ~\cite{arentze2013travelers} for a comprehensive introduction to the topic.

In recent years, the empirical research on Human Mobility has taken new directions. A growing body of literature has focused on quantitative descriptions of human movements from large, automatically collected data sources, such as mobile phone records, travel cards and GPS traces~\cite{barbosa2018human}. In this section, we overview recent empirical findings on multi-modal mobility in the field of Complex Systems, which focused on two important aspects: the dynamics of \emph{public transport systems}, whose study was driven by the availability of public transport data such as schedules and positions of stop and stations; and \emph{individual multi-modal behavior}, driven by the availability of data collected using `smart travel cards' and GPS data. 

The existing empirical research on multi-modal travel based on passively collected data-sources is far from being comprehensive. Most studies have focused on public transit, such that the interplay between public and forms of private transportation such as walking, driving and cycling has been poorly characterized. Further, several studies are based on public transport schedules instead of real-time data, thus neglect important effects deriving from congestion. The increasing availability of high-resolution GPS trajectories collected by individual mobile phones and sensors installed on private and public transport vehicles~\cite{barbosa2018human}, will be key to fill these gaps in the literature.

\subsubsection{Public transport systems dynamics.} 
Over the last decade, the availability of detailed public transport schedules shared by public transport companies has allowed to better estimate travel times and characterize transport systems. 

\paragraph{Efficiency} 
To satisfy the demand of large number of individuals while reducing energy and costs, multilayer transport systems must achieve high efficiency. One aspect concerns the \emph{synchronization} between the network layers, because the more layers are synchronized, the less users have to wait for vehicles. The synchronization inefficiency $\delta(i,j)$~\cite{Gallotti2014Efficiency,barthelemy2016structure} for nodes $i$ and $j$ can be measured as as the ratio of the time-respecting travel time $\tau_t(i,j)$, which accounts for walking and waiting times and the fact that the speed of vehicles varies during the day, and the minimal travel time $\tau_m(i,j)$, where one assumes that vehicles travel at their maximum speed and transfer are instantaneous:

\begin{equation}
    \delta(i,j)=\frac{\tau_t(i,j)}{\tau_m(i,j)}-1
\end{equation}

Using the synchronization inefficiency, it was shown~\cite{Gallotti2014Efficiency} that, on average in the UK, $23\%$ of travel time is lost in connections for trips with more than one mode. Interestingly, across several urban transport system in the UK, the synchronization efficiency $\delta(i,j)$ obeys the same scaling relation with the path length $\ell(i,j)$:

\begin{equation} \label{eq:deltaSynchronization}
    \delta(i,j) \approx \delta_{\textit{min}}+\frac{\delta_{\textit{max}}-\delta_{\textit{min}}}{\ell(i,j)^v},
\end{equation}

with $v \approx 0.5$, and where $\delta_{\textit{max}}$ and $\delta_{\textit{min}}$ are the maximum and minimum values of $\delta(i,j)$ for a given urban transport system.

Further, it has been shown that the average synchronization inefficiency $\overline{\delta}$ for a given urban system follows: 

\begin{equation}
   \delta \sim \Omega^{-\mu},
\end{equation}

where $\Omega$ is the total number of stop-events per hour (e.g. the number of times a vehicle stops), and $\mu = 0.3 \pm 0.1$.

Other studies focused on the efficiency in terms of ability to satisfy users demand. In Alessandretti et al.~\cite{alessandretti2016user}, the authors introduced a method based on non-negative matrix factorization to compare the network of commuting flows and the public transport network. The methodology, applied to various public transport systems in France, showed that, while in Paris the transportation system meets the overall demands, it does not in smaller cities, where even people with access to fast public transportation prefer to use car. In Sui et al.~\cite{sui2019publictransport}, the authors proposed three topological metrics to quantify the interaction between public transport network and passengers’ flow and applied it to study differences between the cities of Chengdu and Qingdao in China. In Hollecsek et al.~\cite{holleczek2014detecting} the authors used data mining approaches to compare the use of public and private transportation and identify the existence of weak transportation connections.
 
\paragraph{Congestion.} 
Congestion can dramatically alter travel time estimates for routes that use popular network links in a urban system, but adding network layers in a multilayer transport system can help reduce global congestion~\cite{chodrow2016congestion}. Given a network with edges $e$ and flows $j_e$, one can quantify the total time lost in congestion as 

\begin{equation}
    T_c(\textbf{j})=\sum_{e\in \mathcal{L}}j_e(t_e^* - t_e(j_e)),
\end{equation}

where $\textbf{j}$ is the vector of flows whose $e$th element is the flow along edge $e$, $t_e^*$ is the free flow time on edge $e$ (in the absence of congestion), and $t_e(j_e)$ is the congested travel time in the presence of flow $j_e$. 
This measures can be used to analyze the impact of changes in flow along a route. The quantity
\begin{equation}
    \Delta_p = - \nabla T_c(\textbf{j}) \cdot \textbf{e}_p
\end{equation}

where $p$ is a path and $\textbf{e}_p$ is the vector whose $e$th component is 1 iff $e \in p$, quantifies the impact of removing a single unit of flow from $p$ on the global congestion function $T_c$. 
Chodrow et al.~\cite{chodrow2016congestion}, quantified how the creation of a planned metro network in Riyadh would affect congestion, by quantifying the change $\Delta_p$ as a function of the speed ratio between the street and metro systems:

\begin{equation}
    \beta = v_{c}/v_{m}
\end{equation}

The authors showed that, as the subway speed increases the global congestion is reduced, but increases locally close to key metro station.

\paragraph{Navigability} 
As cities and their transportation systems become increasingly complex and multimodal, it is important to quantify our difficulty navigating in them. It has been shown that multilayer transport system are characterized by limited navigability, implying that finding one's way is cognitively challenging \cite{gallotti2016limits}. To quantify the difficulty of navigating between two nodes $s$ and $t$ in a network, one can compute the total information value of knowing any of the shortest paths to reach $t$ from $s$:

\begin{equation}
S(s \rightarrow t) = - log_2 \sum_{\{p(s,t)\}} P[p(s,t)]
\end{equation}

where  ${p(s, t)}$ is the set of shortest paths between $s$ and $t$ (note that there can be more than one with the same length) and $P[p(s,t)]$ is the probability to follow path $p(s,t)$, making the right choice at each intersection along the path~\cite{rosvall2005networks}:

\begin{equation}
     P[p(s,t)]=\frac{1}{k_s}\prod_{j\in p(s,t)}\frac{1}{k_j-1}, 
\end{equation}

In Gallotti et al.~\cite{gallotti2016limits} the authors quantified the amount of information an individual needs in order to travel along the shortest path between any given pair of metro stations, in the single-layer metro networks for 15 large cities. They found that this information has an upper bound of the order of 8 bits, corresponding to approximately 250 connections between different routes. Further, studying several among the largest multi-layer transport networks (metro/buses/light rail), they showed that the amount of information one needs to know to travel between any two points exceeds the identified cognitive limit of 8 bits in 80\% the cases, suggesting multi-layer are too complex for individuals to navigate easily.

\subsubsection{Individual multi-modal behaviour.}
In recent years, data collected via smart travel cards has dramatically improved our ability to characterize multimodal behaviour in urban transport systems, overcoming some of the limitations related to collecting and analyzing survey data~\cite{chen2016promises,zannat2019emerging}. Smart-card automated fare collection systems allow passengers to make journeys involving different transport modes using magnetic cards and automatic gate machines. As these systems identify and store the location and time where individuals board and, in some cases, alight public transport, they collect accurate descriptions of individual travel~\cite{pelletier2011smart}. Concurrently, advancements were made possible by the development of methodologies allowing to identify typical travel patterns~\cite{ma2013mining}.


\paragraph{Route choices.} 
Smart-card data has allowed to quantify how individuals navigate multilayer networks. One of the key findings is that individuals do not choose optimal paths (those with shortest travel time), especially when the system is congested. Focusing on the bus and subway trips of 2.4 million passengers in Shenzen (China), Zheng et al.~\cite{zheng2018coupling} studied the coupling (see eq.\ref{eq:coupling}) between the bus and subway layers. In contrast to previous studies~\cite{strano2015features}, the authors characterized the coupling $\lambda$ engendered by passengers' behavior rather than structural properties of the multilayer network. Under their definition, the \emph{coupling} between layers is the fraction of multimodal trips actually undertaken by passengers, rather than the fraction of multimodal shortest paths (see eq.\ref{eq:coupling}). The authors find that this `behavioral' coupling correlates weakly with the empirical speed ratio measured between the two layers over time, implying that passengers choose unimodal trips even when multimodal trips may be preferred because one of the two layers is congested. This finding highlights that the speed ratio of different network layers, which was regarded as a key factor in determining coupling strength~\cite{strano2015features,chodrow2016congestion}, may have a negligible effect on travelers’ route selections, possibly because passengers do not have a full view of the status of traffic. Instead, the authors showed that the coupling between layers is generated by long-distance trips originating from nodes served by a single transport layer.

\paragraph{Power-law distribution of displacements} 
The availability of large-scale data sources has allowed to reveal that individual mobility patterns display universal properties. One key finding is that the distribution $P(\Delta r)$ describing the probability of travelling a given distance $\Delta r$ is characterized by a power-law tail $P(\Delta r)\sim \Delta r ^{-\beta}$, with $1 \leq \beta \leq 2$~\cite{barbosa2018human}. Interestingly, it was shown that this observation could be the effect of the widespread use of multimodal transportation infrastructure~\cite{gallotti2016stochastic,zhao2015explaining}: In Zhao et al.~\cite{zhao2015explaining}, the authors used GPS data to show that mobility using a single mode can be approximated by a lognormal distribution, but the mixture of the distributions associated with each modality generates a power-law. In Gallotti et al.~\cite{gallotti2016stochastic}, the authors found that a simple model where individual trajectories are subject to changes in velocity generates a distribution of displacements with a power-law tail. In fact, individuals using multimodal infrastructure are subject to drastic changes in velocity. In Verga et al.~\cite{varga2016further}, the authors showed that the travel speed $v$ increases with travel distance according to the power-law functional form $v \sim r ^\alpha$, where $\alpha \approx 0.5$ . This dependence is due to the hierarchical structure of transportation systems and the fact that waiting-times (parking, take-off, landing, etc) decrease as a function of trip distance.

\section{Open data \& Tools \label{sec:datatools}}

During the last years there has been an increase in data availability and tools to study multimodal transport networks, in this section we highlight some of the datasets that are available, and computational tools to analyze the data as a complex networks and more specifically as multiplex transport networks.

\subsection{Data}
There are two main types of data available, the one that captures the infrastructure, and the one about dynamics. For the first one the largest open data repository that has been used to analyze transportation networks is OpenStreetMap, a high quality~\cite{haklay2010openstreetmap,girres2010quality,ferster2019openstreetmap,barbosa2018human}, collaborative effort to map the world. For the later, different data sets have been published making use of General Transit Feeds Specification (GTFS), smart cards to access public transportation, and some origin destination studies that took into account the mobility options. 

Using the data from OpenStreetMap it is possible to generate transportation networks, specifically for the single layer case multiple datasets have been publish (for an overview see~\cite{boeing2020multiscale,boeing2020world}). With the same data it is possible to build multimodal transport networks,~\cite{gil2015building} proposed a methodology to extract the different layers from OpenStreetMap data and build a multiplex network linking the layers by the intersections between transport modes. Following this approach we~\cite{natera2019data} made public the multiplex transportation data for fifteen cities in the world including cities from developed countries, like the United States of America, to cities in developing countries like Mexico or Colombia.

In 2015 Gallotti et al.~\cite{gallotti2015temporal} published the multilayer temporal network of public transport in Great Britain, a large data set that contains not only the public transport layers in cities, but also the national level covered by coach, planes and ferries. This dataset has the property to include the temporality of the trips encoded in the links, a link is active during a trip and encodes the travel time as the arriving time to the destination node minus the departure time from the origin node. This dataset also includes a static non-temporal network, in which the weight of each edge is the minimal travel-time.

The use of timetables and transit feeds has been useful to bridge and complement the structure with dynamical data. More recently with the digitalization of transport schedules and timetables by public transit operators new data has been made publicly available, an extensive collection of GTFS datasets is available at \url{https://transitfeeds.com/}. This datasets includes the stops, routes and timetables of public transport in multiple cities and providers around the world. Although this data can be analyzed as single layer of public transportation, it can also be used to create multiplex networks in which each layer is a line or transport provider that are interconnected based on their shared stations, a similar approach that the one followed by Alete et al.\cite{Aleta2017Multilayer}. Kujala et al.~\cite{kujala2018collection} published a collection of 25 cities' public transport networks, containing cities from North America, Europe, and Oceania. The dataset differentiates between public transport modes (tram, subway, rail, bus, ferry, cablecar, gondola, funicular), and also includes the walking layer, enabling the construction of multilayer networks when taking into account different transport modes. 

A recent dataset that contains travel time and distance information for the Helsinki Region has been made available by Tenkanen et al.~\cite{tenkanen2020travel}. In this dataset the authors included the travel time and distance information between all the 250 meters statistical grid centroids for the Helsinki Capital Region. The data includes the travel time matrices for 2013, 2015 and 2018. The dataset is multimodal, the included transport modes are walking, cycling, public transport and private car. The travel times were calculated using a door-to-door principle, making the information between different travel modes and year comparable.

\subsection{Tools}

Multiple computational tools have been developed to work with complex networks, such as Networkx~\cite{hagberg2008networkx}, graph-tool~\cite{peixoto2020graphtool}, and igraph~\cite{csardi2006igraph}, enabling research in multiple topics. 

For the transportation and multimodal analysis different tools have been developed, such as OSMnx~\cite{boeing2017osmnx}, a Python package that downloads and constructs street networks from OpenStreetMap, it can download other infrastructure networks too, allowing to download different layers from a city and build their multimodal transport network. In a similar sense, Pyrosm~\cite{tenkanen2020pyrosm} is a python library for reading data from OpenStreetMap's Protocol Buffer Format files (*.osm.pbf) into GeoDataFrames, making it easy to extract multiple datasets from a single file. The General Transit Feed Spec has played an important role to make available more data than can be used to build the public transport layer in a multimodal transport network. Google~\cite{google2020gtfs} developed a Python library to parse, validate and build GTFS files.

Specifically to work with multiplex and multilayer networks different tools have been developed. De Domenico et al.~\cite{dedomenico2015muxviz} published muxViz a tool to visualize and analyze multilayer networks, this tool provides a Graphical User Interface that lets the user interact and analyze with the network.

MAMMULT, is a collection of algorithms developed by Nicosia et al.~\cite{nicosia2015mammult} for the analyzes and modelling of multilayer networks. The functions included in the collection cover from structural properties, such as node, edge, and layer basic properties, to the analysis of dynamics on multilayer network, such as random walks. Although, the tool was developed to work with general multilayer networks, it can be adapted to work with multimodal transport networks.

multiNetX~\cite{kouvaris2015pattern} is a python package that builds on top of Networkx and extends its capabilities to work with multilayer networks, this allows to create undirected weighted and unweighted networks, analyse the spectral properties of the adjacency or Laplacian matrix, and visualize dynamical processes by coloring the nodes and links in the visualizations. 

Pymnet by Kivela et al.~\cite{kivela2018pymnet} is one of the most recent Python libraries to work with multilayer networks. The library handles general multilayer networks, including multiplex networks with temporal variables, and has multiple network analysis methods, transformations, and models for its analysis and visualization.

The multinet library for the analysis of multilayer networks by Magnani et al.~\cite{magnani2020multiplex}, is a Python and R library that provides tools and algorithms to work with multilayer networks including community detection and visualization methods.

Rivero et al.~\cite{rivero2020algebraic} developed \textit{multiplex}, an R library to work with multilayer networks, apart from manipulating and visualizing multilayer networks, this package offers multiple functions to work with algebraic systems - such as the partially ordered semigroup, and balance or cluster semirings - their decomposition, and the enumeration of bundle patterns occurring at different levels of the network. 