\chapter{Related Work}\label{ch:litReview}

In this chapter we discuss relevant previous work on corruption, including its measurement and its relational aspects, in order to orient the thesis. Following a brief survey of classical studies of corruption, we review how corruption is studied using experiments and models. These results give us insights into how corruption functions at smaller scales and suggests potential mechanisms to investigate. Next comes a review of how corruption is measured, considering surveys, and indicators derived from administrative data. We then survey findings on the causes and consequences of corruption, noting how corruption is measured in each result. With these results in mind, we argue that measuring corruption using indicators derived from administrative data is the most promising way to study the organization and structure of corruption because of its granularity. Correlations between such indicators and other measures of corruption demonstrate the validity of our chosen approach. Finally, we review applications of network methods to the study of corruption and crime more generally.
