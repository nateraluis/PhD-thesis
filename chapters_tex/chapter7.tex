\chapter{Appendices}

\section{Life quality as walkability}

\subsection{Secondary data sources}
\label{SI:walkabilityData}
\begin{itemize}
  \item {Sport associations in Budapest}~\cite{HU_sport}
  \item {Kindergartens, daycares, primary and secondary education}~\cite{HU_Edu}
  \item {Art and music schools}~\cite{HU_Art}
  \item {Child health services}~\cite{HU_Child}
  \item {Social welfare system (eg.: elderly care)}~\cite{HU_Social}
  \item{Culture centers}~\cite{HU_Cult}
  \item {Indoor playgrounds}~\cite{HU_Play}
  \item{Healthcare (hospitals, private and public clinics, specialists)}~\cite{HU_Health}
  \item{Fitness and training facilities}~\cite{HU_Fitness}
  \item{Outdoor fitness facilities}~\cite{HU_outfitness}
  \item{Thermal baths and spa}~\cite{HU_Thermal}
  \item{Playgrounds and parks}~\cite{HU_Park}
\end{itemize}

\subsection{Weights used in the calculations}
The weights of the different $Q$ indices in the final aggregation as well as in sub-categories highly depends on the context and the nature of the problem. Here we present the values we used to generate the results of this study, that were agreed upon consulting with experts.

The weights of the sub-indices from equation (\ref{final_Q}) are of the following values:
\begin{itemize}
  \item $w^{\text{services}}=0.7$;
  \item $w^{\text{safety}}=0.1$;
  \item $w^{\text{environment}}=0.2$
\end{itemize}

The category weights used in equation (\ref{Q_services}), aggregating $Q^{services}$ are:

\begin{itemize}
  \item $w^{\text{family}}= 0.3$;
  \item $w^{\text{health}}= 0.3$;
  \item $w^{\text{culture}} = 0.15$;
  \item $w^{\text{sport}} = 0.15$;
  \item $w^{\text{night life}}=0.1$
\end{itemize}
